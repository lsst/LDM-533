\documentclass[DM,lsstdraft,STS,toc]{lsstdoc}
\usepackage{enumitem}
\usepackage{booktabs}
\input meta.tex

\begin{document}

\providecommand{\tightlist}{%
  \setlength{\itemsep}{0pt}\setlength{\parskip}{0pt}}

\def\product{LSST Level 1 System}

\setDocCompact{true}

\title[Test Spec for \product]{\product~Test Specification}

\author{Eric C. Bellm, John D. Swinbank}
\setDocRef{\lsstDocType-\lsstDocNum}
\setDocDate{\vcsdate}

\setDocAbstract {
This document describes the detailed test specification for the \product{}.
}

% Most recent last
\setDocChangeRecord{%
	\addtohist{1.0}{2018-01-11}{Initial release of draft version.}{Bellm, Swinbank}
	\addtohist{1.1}{2018-01-11}{Implementation of \jira{RFC-429}}{T.~Jenness}
        \addtohist{}{2018-07-13}{Update document with latest test cases defined in Jira}{G.~Comoretto}
}

\setDocCurator{Eric C. Bellm}
\setDocUpstreamLocation{\url{https://github.com/lsst/ldm-533}}
\setDocUpstreamVersion{\vcsrevision}

\maketitle

\section{Introduction}
\label{sec:intro}

This document specifies the test procedure for the \product{}.

The \product{} is the component of the LSST system which is responsible for
scientific processing leading to:

\begin{itemize}

  \item{Single frame processing and measurement;}
  \item{Alert generation from difference image analysis;}
  \item{Alert distribution to community brokers;}
  \item{Simple filtering of alerts;}
  \item{Precovery and forced photometry measurements on new and previously-known
	  sources found in difference imaging;}
  \item{Identification of moving objects.}
  \item{Generating QC metrics based on pipeline execution and post-processing of
  scientific data products.}


\end{itemize}

A full description of this product is provided in \S6 (which describes the
Data Facility-provided execution services) and \S13.1 (the science payloads) of
\citeds{LDM-148}.

\subsection{Objectives}
\label{sec:objectives}

This document builds on the description of LSST Data Management's approach to
testing as described in \citeds{LDM-503} to describe the detailed tests that
will be performed on the \product{} as part of the verification of the DM system.

It identifies test designs, test cases and procedures for the tests, and the
pass/fail criteria for each test.

\subsection{Scope}
\label{sec:scope}

This document describes the test procedures for the following components of
the LSST system (as described in \citeds{LDM-148}):

\begin{itemize}

  \item{Services provided by the LSST Data Facility:

    \begin{itemize}
      \item{Prompt Processing Execution}
      \item{Batch and Offline Processing Execution}
      \item{Level 1 Quality Control}
      \item{Alert Distribution Execution}
      \item{Alert Filtering Execution}
    \end{itemize}
  }

  \item{Science payloads:

    \begin{itemize}
      \item{Single frame processing Payload}
      \item{Alert Generation Payload}
      \item{Precovery and Forced Photometry Payload}
      \item{MOPS Payload}
    \end{itemize}

  }

\end{itemize}

\subsection{Applicable Documents}
\label{sec:docs}

\addtocounter{table}{-1}

\begin{tabular}[htb]{l l}
\citeds{LDM-148} & LSST DM System Architecture \\
\citeds{LDM-151} & LSST DM Science Pipelines Design \\
\citeds{LDM-294} & LSST DM Organization \& Management \\
\citeds{LDM-502} & The Measurement and Verification of DM Key Performance Metrics \\
\citeds{LDM-503} & LSST DM Test Plan \\
\citeds{LSE-61}  & LSST DM Subsystem Requirements \\
\citeds{LSE-163} & LSST Data Products Definition Document \\
\end{tabular}

\subsection{References\label{sect:references}}
\renewcommand{\refname}{}
\bibliography{lsst,refs,books,refs_ads}

%\subsection{Definitions, acronyms, and abbreviations \label{sect:acronyms}} % include acronyms.tex generated by the acronyms.csh (GaiaTools)
%\input{acronyms}


%----------------------------------------------------
% TASK IDENTIFICATION - APPROACH
%----------------------------------------------------
\section{Approach}
\label{sec:approach}

The major activities to be performed are to:

\begin{itemize}

  \item{Compare the design of the Alert Production payload as
  implemented to the requirements on the outputs of the DM Subsystem as
  defined in \citeds{LSE-63} and \citeds{LSE-163} to demonstrate that all data
  products required by the scientific community will be delivered by the
  system as built.}

  \item{Ensure that all data products included in the AP payload design are
  correctly produced and persisted appropriately to the LSST
  Data Backbone, Alert Distribution System, and/or Alert Filtering service
  as appropriate.}

  \item{Ensure that all data products required by the Precovery and Forced
  Photometry payload are correctly
  produced and persisted appropriately to the LSST Data Backbone.}

  \item{Ensure that all data products required by the MOPS system are correctly
  produced and persisted appropriately to the LSST Data Backbone.}

  \item{Demonstrate that QC metrics are properly calculated and transmitted
  during the execution all L1 production types.}

  \item{Demonstrate that post-processing QC analysis of data products can be
  used to identify and report on failures or anomalies in the processing.}

\end{itemize}

\subsection{Tasks and criteria}
\label{sec:tasks}

The following are the major items under test:

\begin{itemize}

  \item{The science payload capable of prompt processing of single visit
	  images;}

  \item{The Alert Generation payload that detects variable sources through
	  difference image analysis;}

  \item{The Alert Distribution System that packages alerts and forwards
	  them to community brokers;}

  \item{The filtering system that allows science users to apply simple
	  filters to the alert stream;}


  \item{The Precovery and Forced Photometry payloads that measure flux
	  levels for new and previously-known sources found in difference
	  images;}

  \item{The Moving Object Processing System payload that identifies solar
	  system bodies from difference image sources;}

  \item{Services capable of scheduling and managing the execution of all of
  the above payloads, marshalling their results, and making them available to
  other parts of the system for analysis or further distribution.}

\end{itemize}

\subsection{Features to be tested}
\label{sec:feat2test}

\begin{itemize}

  \item{Code for the payloads described in \S\ref{sec:tasks} can be made available on systems managed by the LSST Data Facility;}
  \item{The payloads described in \S\ref{sec:tasks} can be executed under the control of appropriate LDF services;}
  \item{All required data products are persisted;}
  \item{Data products exhibit scientific fidelity, satisfying the requirements described in \citeds{LSE-61}.}

\end{itemize}

\subsection{Features not to be tested}
\label{sec:featnot2test}

This test specification does not extend to demonstrating the detailed compliance of LSST data products with all \citeds[Science Requirements Document]{LPM-17} level requirements: 
such a demonstration would require carefully curated LSST-like datasets (or simulated data), a detailed understanding of the precursor observing system at the level required by LSST, LSST-like calibration products, etc., which are assumed not to be available for all test cases.

This document does not describe facilities for periodically generating or
collecting key performance metrics (KPMs), except insofar as those KPMs are
incidentally measured as part of executing the documented test cases. The KPMs
and the system being used to track KPMs and to ensure compliance with
documented requirements is described in \citeds{LDM-502}.

\subsection{Pass/fail criteria}
\label{sec:passfail}

The results of all tests will be assessed using the criteria described in
\citeds{LDM-503} \S4.

Note that, when executing pipelines, tasks or individual algorithms, any
unexplained or unexpected errors or warnings appearing in the associated log
or on screen output must be described in the documentation for the system
under test. Any warning or error for which this is not the case must be filed
as a software problem report and filed with the DMCCB.

\subsection{Suspension criteria and resumption requirements}
\label{suspension}

Refer to individual test cases where applicable.

\subsection{Naming convention}

With the introduction of the Jira Test Management, the following definitions have to be considered:

\begin{description}
  \item[LVV]{: Is the label for the ``LSST Verification and Validation'' project in Jira where all information regarding tests are managed.}
  \item[LVV-XXX]{: Are Verification Elements, where XXX is the Verification Element identifier.  Each Verification Element is derived from a requirement and has at least one Test Case associated. There can be multiple Verification Elements associated with a requirement.}
  \item[LVV-TYYY]{: Are Test Cases. Each Test Case is associated with a Verification Element, where YYY is the Test Case identifier. There can be multiple test cases associated with a Verification Element.}
\end{description}

The old naming convention described below is now obsolete, but existing test cases can be encountered in old documents, and will keep the original identifier at the beginning of the name.

Tests were named according to the pattern \textsc{prod-xx-yy} where:

\begin{description}[font=\normalfont\scshape]

  \item[prod]{The product under test. Relevant entries for this document are:
    \begin{description}[font=\normalfont\scshape,topsep=-1.0ex]
      \item[AG]{The Alert Generation payload and associated service}
    \end{description}
  }
  \item[xx]{Test specification number (in increments of 10)}
  \item[yy]{Test case number (in increments of 5)}

\end{description}

\subsection{Test Cases Summary}\label{test-cases-summary}

\begin{longtable}[]{p{3cm}p{13cm}}
\toprule
Jira Id & Test Name\tabularnewline
\midrule
\endhead
\protect\hyperlink{lvv-t17---ag-00-00-installation-of-the-alert-generation-science-payload}{LVV-T17}
&
\href{https://jira.lsstcorp.org/secure/Tests.jspa\#/testCase/LVV-T17}{AG-00-00:
Installation of the Alert Generation science payload.}\tabularnewline
\protect\hyperlink{lvv-t18---ag-00-05-alert-generation-produces-required-data-products}{LVV-T18}
&
\href{https://jira.lsstcorp.org/secure/Tests.jspa\#/testCase/LVV-T18}{AG-00-05:
Alert Generation Produces Required Data Products}\tabularnewline
\protect\hyperlink{lvv-t19---ag-00-10-scientific-verification-of-processed-visit-images}{LVV-T19}
&
\href{https://jira.lsstcorp.org/secure/Tests.jspa\#/testCase/LVV-T19}{AG-00-10:
Scientific Verification of Processed Visit Images}\tabularnewline
\protect\hyperlink{lvv-t20---ag-00-15-scientific-verification-of-difference-images}{LVV-T20}
&
\href{https://jira.lsstcorp.org/secure/Tests.jspa\#/testCase/LVV-T20}{AG-00-15:
Scientific Verification of Difference Images}\tabularnewline
\protect\hyperlink{lvv-t21---ag-00-20-scientific-verification-of-diasource-catalog}{LVV-T21}
&
\href{https://jira.lsstcorp.org/secure/Tests.jspa\#/testCase/LVV-T21}{AG-00-20:
Scientific Verification of DIASource Catalog}\tabularnewline
\protect\hyperlink{lvv-t22---ag-00-25-scientific-verification-of-diaobject-catalog}{LVV-T22}
&
\href{https://jira.lsstcorp.org/secure/Tests.jspa\#/testCase/LVV-T22}{AG-00-25:
Scientific Verification of DIAObject Catalog}\tabularnewline
\protect\hyperlink{lvv-t216---ad-00-00-installation-of-the-alert-distribution-payloads}{LVV-T216}
&
\href{https://jira.lsstcorp.org/secure/Tests.jspa\#/testCase/LVV-T216}{AD-00-00:
Installation of the Alert Distribution payloads.}\tabularnewline
\protect\hyperlink{lvv-t217---ad-00-05-full-stream-alert-distribution}{LVV-T217}
&
\href{https://jira.lsstcorp.org/secure/Tests.jspa\#/testCase/LVV-T217}{AD-00-05:
Full Stream Alert Distribution}\tabularnewline
\protect\hyperlink{lvv-t218---ad-00-10-simple-filtering-of-the-lsst-alert-stream}{LVV-T218}
&
\href{https://jira.lsstcorp.org/secure/Tests.jspa\#/testCase/LVV-T218}{AD-00-10:
Simple Filtering of the LSST Alert Stream}\tabularnewline
\bottomrule
\end{longtable}

\section{Test Cases}\label{test-cases}

\subsection{\texorpdfstring{\href{https://jira.lsstcorp.org/secure/Tests.jspa\#/testCase/LVV-T17}{LVV-T17}
- AG-00-00: Installation of the Alert Generation science
payload.}{LVV-T17 - AG-00-00: Installation of the Alert Generation science payload.}}\label{lvv-t17---ag-00-00-installation-of-the-alert-generation-science-payload.}

\begin{longtable}[]{@{}llllll@{}}
\toprule
Version & Status & Priority & Verification Type & Critical Event &
Owner\tabularnewline
\midrule
\endhead
1 & Approved & Normal & Test & False & Eric Bellm\tabularnewline
\bottomrule
\end{longtable}

\subsubsection{Test Items}\label{test-items}

This test will check:

\begin{itemize}
\tightlist
\item
  That the Alert Generation science payload is available for
  distribution from documented channels;
\item
  That the Alert Generation science payload can be installed on LSST
  Data Facility-managed systems.
\end{itemize}

\subsubsection{Requirements}\label{requirements}

\begin{itemize}
\tightlist
\item
  \href{https://jira.lsstcorp.org/browse/LVV-139}{LVV-139} -
  DMS-REQ-0308-V-01: Software Architecture to Enable Community Re-Use
\end{itemize}

\subsubsection{Precondition}\label{precondition}

- Input specification\\[2\baselineskip]No input data is required for
this test case.

\subsubsection{Test Script}\label{test-script}

\textbf{Step 1}\\
Release 16.0 of the LSST Science Pipelines will be installed into the
GPFS filesystem accessible at /software on lsst-dev01 following the
instructions at \url{https://pipelines.lsst.io/install/newinstall.html}
.\\[2\baselineskip]\textbf{Step 2}\\
The lsst\_distrib top level package will be
enabled:\\[2\baselineskip]\hspace*{0.333em} ~ ~ ~source
/software/lsstsw/stack3/loadLSST.bash\\
\hspace*{0.333em} ~ ~ ~setup lsst\_distrib\\[2\baselineskip]\textbf{Step
3}\\
The ``LSST Stack Demo'' package will be downloaded onto the test system
from
\href{https://github.com/lsst/lsst_dm_stack_demo/releases/tag/14.0}{https://github.com/lsst/lsst\_dm\_stack\_demo/releases/tag/16.0}
and uncompressed.\\[2\baselineskip]\textbf{Step 4}\\
The demo package will be executed by following the instructions in its
``README`` file. The string ``Ok.`` should be returned. Specifically, we
execute:\\
\hspace*{0.333em} ~ ~ ~setup obs\_sdss\\
\hspace*{0.333em} ~ ~ ~./bin/demo.sh\\
\hspace*{0.333em} ~ ~ ~python bin/compare
expected/Linux64/detected-sources.txt\\[2\baselineskip]\textbf{Step 5}\\
A shell on an LSST-VC compute node will now be obtained by executing:\\
\hspace*{0.333em} ~ ~\$ srun -I --pty bash\\[2\baselineskip]\textbf{Step
6}\\
The demo package will be executed on the compute node and the same
result obtained.\\[2\baselineskip]\textbf{Step 7}\\
The Alert Production datasets and packages are not yet part of
lsst\_distrib and so must be installed separately. They will be
installed as follows on the GPFS
filesystem:\\[2\baselineskip]\hspace*{0.333em} ~ setup git\_lfs\\
\hspace*{0.333em} ~ git clone
https://github.com/lsst/ap\_verify\_hits2015.git\\[2\baselineskip]\hspace*{0.333em}
~ export AP\_VERIFY\_HITS2015\_DIR=\$PWD/ap\_verify\_hits2015 cd
\$AP\_VERIFY\_HITS2015\_DIR\\
\hspace*{0.333em} ~ setup -r .\\
\hspace*{0.333em} ~ cd-\\
\hspace*{0.333em} ~\\
\hspace*{0.333em} ~ setup obs\_decam\\
\hspace*{0.333em} ~ git clone
https://github.com/lsst-dm/ap\_association\\
\hspace*{0.333em} ~ cd ap\_association\\
\hspace*{0.333em} ~ setup -k -r .\\
\hspace*{0.333em} ~ scons\\
\hspace*{0.333em} ~ cd-\\
\hspace*{0.333em} ~\\
\hspace*{0.333em} ~ git clone https://github.com/lsst-dm/ap\_pipe\\
\hspace*{0.333em} ~ cd ap\_pipe\\
\hspace*{0.333em} ~ setup -k -r .\\
\hspace*{0.333em} ~ scons\\
\hspace*{0.333em} ~ cd-\\
\hspace*{0.333em} ~\\
\hspace*{0.333em} ~ git clone https://github.com/lsst-dm/ap\_verify\\
\hspace*{0.333em} ~ cd ap\_verify\\
\hspace*{0.333em} ~ setup -k -r .\\
\hspace*{0.333em} ~ scons\\
\hspace*{0.333em} ~ cd-\\[2\baselineskip]and any errors or failures
reported.\\[2\baselineskip]

\hypertarget{lvv-t18---ag-00-05-alert-generation-produces-required-data-products}{\subsection{\texorpdfstring{\href{https://jira.lsstcorp.org/secure/Tests.jspa\#/testCase/LVV-T18}{LVV-T18}
- AG-00-05: Alert Generation Produces Required Data
Products}{LVV-T18 - AG-00-05: Alert Generation Produces Required Data Products}}\label{lvv-t18---ag-00-05-alert-generation-produces-required-data-products}}

\begin{longtable}[]{@{}llllll@{}}
\toprule
Version & Status & Priority & Verification Type & Critical Event &
Owner\tabularnewline
\midrule
\endhead
1 & Approved & Normal & Test & False & Eric Bellm\tabularnewline
\bottomrule
\end{longtable}

\subsubsection{Test Items}\label{test-items-1}

This test will check that the basic data products produced by Alert
Generation are generated by execution of the science payload.\\
These products will include:

\begin{itemize}
\tightlist
\item
  Processed visit images (PVIs; DMS-REQ-0069); • Difference Exposures
  (DMS-REQ-0010);
\item
  DIASource catalogs (DMS-REQ-0269);
\item
  DIAObject catalogs (DMS-REQ-0271);
\end{itemize}

\subsubsection{Requirements}\label{requirements-1}

\begin{itemize}
\tightlist
\item
  \href{https://jira.lsstcorp.org/browse/LVV-29}{LVV-29} -
  DMS-REQ-0069-V-01: Processed Visit Images
\item
  \href{https://jira.lsstcorp.org/browse/LVV-7}{LVV-7} -
  DMS-REQ-0010-V-01: Difference Exposures
\item
  \href{https://jira.lsstcorp.org/browse/LVV-100}{LVV-100} -
  DMS-REQ-0269-V-01: DIASource Catalog
\item
  \href{https://jira.lsstcorp.org/browse/LVV-102}{LVV-102} -
  DMS-REQ-0271-V-01: DIAObject Catalog
\end{itemize}

\subsubsection{Precondition}\label{precondition-1}

- Input specification\\[2\baselineskip]A complete processing of the
DECam ``HiTS'' dataset, as defined at https://dmtn-039.lsst. io/ and
https://github.com/lsst/ap\_verify\_hits2015, through the Alert
Generation science payload.\\
This dataset shall be made available in a standard LSST data repository,
accessible via the ``Data Butler''.\\
It is not required that all combinations of visit and CCD have been
processed successfully: a number of failures are expected. However,
documentation to describe processing failures should be provided.

\subsubsection{Test Script}\label{test-script-1}

\textbf{Step 1}\\
The DM Stack and Alert Processing packaged shall be initialized as
described in LVT-T17 (AG-00-00).\\[2\baselineskip]\textbf{Step 2}\\
The alert generation processing will be executed using the verification
cluster:\\[2\baselineskip]```bash\\
python ap\_verify/bin/prepare\_demo\_slurm\_files.py\\
\# At present we must run a single ccd+visit to handle ingestion
before\\
\# parallel processing can begin\\
./ap\_verify/bin/exec\_demo\_run\_1ccd.sh 410915 25\\
ln -s ap\_verify/bin/demo\_run.sl\\
ln -s ap\_verify/bin/demo\_cmds.conf\\
sbatch demo\_run.sl\\
```\\[2\baselineskip]and any errors or failures
reported.\\[2\baselineskip]\textbf{Step 3}\\
A ``Data Butler'' will be initialized to access the
repository.\\[2\baselineskip]\textbf{Step 4}\\
For each of the expected data products types (listed in §4.2.2) and each
of the expected units (PVIs, catalogs, etc.), the data product will be
retrieved from the Butler and verified to be
non-empty.\\[2\baselineskip]\textbf{Step 5}\\
DIAObjects are currently only stored in a database, without shims to the
Butler, so the existence of the database table and its non-empty
contents will be verified by directly accessing it using sqlite3 and
executing appropriate SQL queries.\\[2\baselineskip]

\hypertarget{lvv-t19---ag-00-10-scientific-verification-of-processed-visit-images}{\subsection{\texorpdfstring{\href{https://jira.lsstcorp.org/secure/Tests.jspa\#/testCase/LVV-T19}{LVV-T19}
- AG-00-10: Scientific Verification of Processed Visit
Images}{LVV-T19 - AG-00-10: Scientific Verification of Processed Visit Images}}\label{lvv-t19---ag-00-10-scientific-verification-of-processed-visit-images}}

\begin{longtable}[]{@{}llllll@{}}
\toprule
Version & Status & Priority & Verification Type & Critical Event &
Owner\tabularnewline
\midrule
\endhead
1 & Approved & Normal & Test & False & Eric Bellm\tabularnewline
\bottomrule
\end{longtable}

\subsubsection{Test Items}\label{test-items-2}

This test will check that the Processed Visit Images (PVIs) delivered by
the alert generation science payload meet the requirements laid down by
LSE-61.\\
Specifically, this will demonstrate that:

\begin{itemize}
\tightlist
\item
  Processed visit images have been generated and persisted during
  payload execution;
\item
  Each PVI includes a science pixel array, a mask array, and a variance
  array. (DMS-REQ- 0072).
\item
  Each PVI includes a background model (DMS-REQ-0327), photometric
  zero-point (DMS- REQ-0029), spatially-varying PSF (DMS-REQ-0070) and
  WCS (DMS-REQ-0030).
\item
  Saturated pixels are correctly masked.
\item
  Pixels affected by cosmic rays are correctly masked.
\item
  The background is not oversubtracted around bright objects.
\end{itemize}

This test does not include quantitative targets for the science quality
criteria.

\subsubsection{Requirements}\label{requirements-2}

\begin{itemize}
\tightlist
\item
  \href{https://jira.lsstcorp.org/browse/LVV-29}{LVV-29} -
  DMS-REQ-0069-V-01: Processed Visit Images
\item
  \href{https://jira.lsstcorp.org/browse/LVV-158}{LVV-158} -
  DMS-REQ-0327-V-01: Background Model Calculation
\item
  \href{https://jira.lsstcorp.org/browse/LVV-12}{LVV-12} -
  DMS-REQ-0029-V-01: Generate Photometric Zeropoint for Visit Image
\item
  \href{https://jira.lsstcorp.org/browse/LVV-30}{LVV-30} -
  DMS-REQ-0070-V-01: Generate PSF for Visit Images
\item
  \href{https://jira.lsstcorp.org/browse/LVV-13}{LVV-13} -
  DMS-REQ-0030-V-01: Generate WCS for Visit Images
\item
  \href{https://jira.lsstcorp.org/browse/LVV-31}{LVV-31} -
  DMS-REQ-0072-V-01: Processed Visit Image Content
\end{itemize}

\subsubsection{Precondition}\label{precondition-2}

\textbf{Input specification}\\[2\baselineskip]A complete processing of
the DECam ``HiTS'' dataset, as defined at https://dmtn-039.lsst.io/ and
https://github.com/lsst/ap\_verify\_hits2015, through the Alert
Generation science payload.\\
This dataset shall be made available in a standard LSST data repository,
accessible via the ``Data Butler''.\\
It is not required that all combinations of visit and CCD have been
processed successfully: a number of failures are expected. However,
documentation to describe processing failures should be provided.

\subsubsection{Test Script}\label{test-script-2}

\textbf{Step 1}\\
The DM Stack shall be initialized using the loadLSST script (as
described in LVV-T17 - AG-00-00).\\[2\baselineskip]\textbf{Step 2}\\
A ``Data Butler'' will be initialized to access the
repository.\\[2\baselineskip]\textbf{Step 3}\\
For each processed CCD, the PVI will be retrieved from the Butler, and
the existence of all components described in §4.3.2 will be
verified.\\[2\baselineskip]\textbf{Step 4}\\
Five sensors will be chosen at random from each of two visits and
inspected by eye for unmasked artifacts.\\[2\baselineskip]

\hypertarget{lvv-t20---ag-00-15-scientific-verification-of-difference-images}{\subsection{\texorpdfstring{\href{https://jira.lsstcorp.org/secure/Tests.jspa\#/testCase/LVV-T20}{LVV-T20}
- AG-00-15: Scientific Verification of Difference
Images}{LVV-T20 - AG-00-15: Scientific Verification of Difference Images}}\label{lvv-t20---ag-00-15-scientific-verification-of-difference-images}}

\begin{longtable}[]{@{}llllll@{}}
\toprule
Version & Status & Priority & Verification Type & Critical Event &
Owner\tabularnewline
\midrule
\endhead
1 & Approved & Normal & Test & False & Eric Bellm\tabularnewline
\bottomrule
\end{longtable}

\subsubsection{Test Items}\label{test-items-3}

This test will check that the difference images delivered by the Alert
Generation science pay- load meet the requirements laid down by
LSE-61.\\
Specifically, this will demonstrate that:

\begin{itemize}
\tightlist
\item
  Difference images have been generated and persisted during payload
  execution;
\item
  Each difference image includes information about the identity of the
  input exposures, and metadata such as a representation of the PSF
  matching kernel (DMS-REQ-0074);
\item
  Masks are correctly propagated from the input images.
\end{itemize}

This test does not include quantitative targets for the science quality
criteria.

\subsubsection{Requirements}\label{requirements-3}

\begin{itemize}
\tightlist
\item
  \href{https://jira.lsstcorp.org/browse/LVV-7}{LVV-7} -
  DMS-REQ-0010-V-01: Difference Exposures
\item
  \href{https://jira.lsstcorp.org/browse/LVV-32}{LVV-32} -
  DMS-REQ-0074-V-01: Difference Exposure Attributes
\end{itemize}

\subsubsection{Precondition}\label{precondition-3}

- Input specification\\[2\baselineskip]A complete processing of the
DECam ``HiTS'' dataset, as defined at https://dmtn-039.lsst. io/ and
https://github.com/lsst/ap\_verify\_hits2015, through the Alert
Generation science payload.\\
This dataset shall be made available in a standard LSST data repository,
accessible via the ``Data Butler''.\\
It is not required that all combinations of visit and CCD have been
processed successfully: a number of failures are expected. However,
documentation to describe processing failures should be provided.

\subsubsection{Test Script}\label{test-script-3}

\textbf{Step 1}\\
The DM Stack shall be initialized using the loadLSST script (as
described in LVV-T-17 AG-00-00).\\[2\baselineskip]\textbf{Step 2}\\
A ``Data Butler'' will be initialized to access the
repository.\\[2\baselineskip]\textbf{Step 3}\\
For each processed CCD, the difference image will be retrieved from the
Butler, and the existence of all components described in §4.4.2 will be
verified.\\[2\baselineskip]\textbf{Step 4}\\
Five sensors will be chosen at random from each of two visits and the
masks of the input and difference images compared by
eye.\\[2\baselineskip]

\hypertarget{lvv-t21---ag-00-20-scientific-verification-of-diasource-catalog}{\subsection{\texorpdfstring{\href{https://jira.lsstcorp.org/secure/Tests.jspa\#/testCase/LVV-T21}{LVV-T21}
- AG-00-20: Scientific Verification of DIASource
Catalog}{LVV-T21 - AG-00-20: Scientific Verification of DIASource Catalog}}\label{lvv-t21---ag-00-20-scientific-verification-of-diasource-catalog}}

\begin{longtable}[]{@{}llllll@{}}
\toprule
Version & Status & Priority & Verification Type & Critical Event &
Owner\tabularnewline
\midrule
\endhead
1 & Approved & Normal & Test & False & Eric Bellm\tabularnewline
\bottomrule
\end{longtable}

\subsubsection{Test Items}\label{test-items-4}

This test will check that the difference image source catalogs delivered
by the Alert Generation science payload meet the requirements laid down
by LSE-61.

\begin{itemize}
\tightlist
\item
  Specifically, this will demonstrate that:
\item
  Measurements in the catalog are presented in flux units
  (DMS-REQ-0347);
\item
  Each DIASource record contains an appropriate subset of the attributes
  required by DMS-REQ-0269. In particular, the LDM-503-3-era pipeline is
  expected to provide DIA- Source positions (sky and focal plane),
  fluxes, and flags indicative of issues encountered during processing.
\item
  Faint DIASources satisfying additional criteria are stored
  (DMS-REQ-0270).
\item
  Derived quantities are provided in pre-computed columns
  (DMS-REQ-0331);
\end{itemize}

This test does not include quantitative targets for the science quality
criteria.\\[2\baselineskip]

\subsubsection{Requirements}\label{requirements-4}

\begin{itemize}
\tightlist
\item
  \href{https://jira.lsstcorp.org/browse/LVV-100}{LVV-100} -
  DMS-REQ-0269-V-01: DIASource Catalog
\item
  \href{https://jira.lsstcorp.org/browse/LVV-101}{LVV-101} -
  DMS-REQ-0270-V-01: Faint DIASource Measurements
\item
  \href{https://jira.lsstcorp.org/browse/LVV-178}{LVV-178} -
  DMS-REQ-0347-V-01: Measurements in catalogs
\item
  \href{https://jira.lsstcorp.org/browse/LVV-162}{LVV-162} -
  DMS-REQ-0331-V-01: Computing Derived Quantities
\end{itemize}

\subsubsection{Precondition}\label{precondition-4}

- Input specification\\[2\baselineskip]A complete processing of the
DECam ``HiTS'' dataset, as defined at https://dmtn-039.lsst. io/ and
https://github.com/lsst/ap\_verify\_hits2015, through the Alert
Generation science payload.\\
This dataset shall be made available in a standard LSST data repository,
accessible via the ``Data Butler''.\\
It is not required that all combinations of visit and CCD have been
processed successfully: a number of failures are expected. However,
documentation to describe processing failures should be provided.

\subsubsection{Test Script}\label{test-script-4}

\textbf{Step 1}\\
The DM Stack shall be initialized using the loadLSST script (as
described in LVV-T17 - AG-00-00).\\[2\baselineskip]\textbf{Step 2}\\
A ``Data Butler'' will be initialized to access the
repository.\\[2\baselineskip]\textbf{Step 3}\\
DIASource records will be accessed by querying the Butler, then examined
interactively at a Python prompt.\\[2\baselineskip]

\hypertarget{lvv-t22---ag-00-25-scientific-verification-of-diaobject-catalog}{\subsection{\texorpdfstring{\href{https://jira.lsstcorp.org/secure/Tests.jspa\#/testCase/LVV-T22}{LVV-T22}
- AG-00-25: Scientific Verification of DIAObject
Catalog}{LVV-T22 - AG-00-25: Scientific Verification of DIAObject Catalog}}\label{lvv-t22---ag-00-25-scientific-verification-of-diaobject-catalog}}

\begin{longtable}[]{@{}llllll@{}}
\toprule
Version & Status & Priority & Verification Type & Critical Event &
Owner\tabularnewline
\midrule
\endhead
1 & Approved & Normal & Test & False & Eric Bellm\tabularnewline
\bottomrule
\end{longtable}

\subsubsection{Test Items}\label{test-items-5}

This test will check that the DIAObject catalogs delivered by the Alert
Generation science pay- load meet the requirements laid down by
LSE-61.\\
Specifically, this will demonstrate that:

\begin{itemize}
\tightlist
\item
  DIAObjects are recorded with unique identifiers (DMS-REQ-0271);
\item
  Measurements in the catalog are presented in flux units
  (DMS-REQ-0347);
\item
  EachDIAObjectrecordcontainscontainsanappropriatesetofsummaryattributes(DMS-
  REQ-0271 and DMS-REQ-0272). Note:

  \begin{itemize}
  \tightlist
  \item
    This test is executed independently of the Data Release Production
    system. Hence, DIAObjects are not associated to Objects, and the
    association metadata specified by DMS-REQ-0271 is not expected to be
    available.
  \item
    TheLDM-503-3erapipelineisnotexpectedtocalculateorpersistallattributesspec-
    ified by DMS-REQ-0272 requirement.
  \end{itemize}
\item
  Relevant derived quantities are provided in pre-computed columns
  (DMS-REQ-0331);~
\end{itemize}

This test does not include quantitative targets for the science quality
criteria.

\subsubsection{Requirements}\label{requirements-5}

\begin{itemize}
\tightlist
\item
  \href{https://jira.lsstcorp.org/browse/LVV-116}{LVV-116} -
  DMS-REQ-0285-V-01: Level 1 Source Association
\item
  \href{https://jira.lsstcorp.org/browse/LVV-102}{LVV-102} -
  DMS-REQ-0271-V-01: DIAObject Catalog
\item
  \href{https://jira.lsstcorp.org/browse/LVV-103}{LVV-103} -
  DMS-REQ-0272-V-01: DIAObject Attributes
\item
  \href{https://jira.lsstcorp.org/browse/LVV-178}{LVV-178} -
  DMS-REQ-0347-V-01: Measurements in catalogs
\item
  \href{https://jira.lsstcorp.org/browse/LVV-162}{LVV-162} -
  DMS-REQ-0331-V-01: Computing Derived Quantities
\end{itemize}

\subsubsection{Precondition}\label{precondition-5}

\textbf{Input specification}\\[2\baselineskip]A complete processing of
the DECam ``HiTS'' dataset, as defined at https://dmtn-039.lsst. io/ and
https://github.com/lsst/ap\_verify\_hits2015, through the Alert
Generation science payload.\\
This dataset shall be made available in a standard LSST data repository,
accessible via the ``Data Butler''.\\
It is not required that all combinations of visit and CCD have been
processed successfully: a number of failures are expected. However,
documentation to describe processing failures should be provided.

\subsubsection{Test Script}\label{test-script-5}

\textbf{Step 1}\\
The DM Stack shall be initialized using the loadLSST script (as
described in LVV-T17 - AG-00-00).\\[2\baselineskip]\textbf{Step 2}\\
sqlite3 or Python's sqlalchemy module will be used to access the Level 1
database.\\[2\baselineskip]

\subsection{\texorpdfstring{\href{https://jira.lsstcorp.org/secure/Tests.jspa\#/testCase/LVV-T216}{LVV-T216}
- AD-00-00: Installation of the Alert Distribution
payloads.}{LVV-T216 - AD-00-00: Installation of the Alert Distribution payloads.}}\label{lvv-t216---ad-00-00-installation-of-the-alert-distribution-payloads.}

\begin{longtable}[]{@{}llllll@{}}
\toprule
Version & Status & Priority & Verification Type & Critical Event &
Owner\tabularnewline
\midrule
\endhead
1 & Draft & Normal & Test & False & Eric Bellm\tabularnewline
\bottomrule
\end{longtable}

\subsubsection{Test Items}\label{test-items-6}

This test will check:\\[2\baselineskip]

\begin{itemize}
\tightlist
\item
  That the Alert Distribution payloads are available from documented
  channels.
\item
  That the Alert Distribution payloads can be installed on LSST Data
  Facility-managed systems.
\item
  That the Alert Distribution payloads can be executed by LSST Data
  Facility-managed systems.
\end{itemize}

\subsubsection{Requirements}\label{requirements-6}

\begin{itemize}
\tightlist
\item
  \href{https://jira.lsstcorp.org/browse/LVV-139}{LVV-139} -
  DMS-REQ-0308-V-01: Software Architecture to Enable Community Re-Use
\end{itemize}

\subsubsection{Test Script}\label{test-script-6}

\textbf{Step 1}\\
Download Kafka Docker image from
https://github.com/lsst-dm/alert\textbackslash{}\_stream.\\[2\baselineskip]\textbf{Step
2}\\
Change to the alert\_stream directory and build the docker image.\\

\begin{verbatim}
docker build -t "lsst-kub001:5000/alert_stream" .
\end{verbatim}

\textbf{Step 3}\\
Register it with Kubernetes\\[2\baselineskip]docker push
lsst-kub001:5000/alert\_stream\\[2\baselineskip]\textbf{Step 4}\\
From the alert\_stream/kubernetes directory, start Kafka and
Zookeeper:\\[2\baselineskip]

\begin{verbatim}
kubectl create -f zookeeper-service.yaml
kubectl create -f zookeeper-deployment.yaml
kubectl create -f kafka-deployment.yaml
kubectl create -f kafka-service.yaml
\end{verbatim}

(use kubectl get pods/services between each command to check status;
wait until each is ``Running'' before starting the next
command)\\[4\baselineskip]\textbf{Step 5}\\
Confirm Kafka and Zookeeper are listed when
running\\[2\baselineskip]kubectl get
pods\\[2\baselineskip]and\\[2\baselineskip]kubectl get
services\\[2\baselineskip]

\hypertarget{lvv-t217---ad-00-05-full-stream-alert-distribution}{\subsection{\texorpdfstring{\href{https://jira.lsstcorp.org/secure/Tests.jspa\#/testCase/LVV-T217}{LVV-T217}
- AD-00-05: Full Stream Alert
Distribution}{LVV-T217 - AD-00-05: Full Stream Alert Distribution}}\label{lvv-t217---ad-00-05-full-stream-alert-distribution}}

\begin{longtable}[]{@{}llllll@{}}
\toprule
Version & Status & Priority & Verification Type & Critical Event &
Owner\tabularnewline
\midrule
\endhead
1 & Draft & Normal & Test & False & Eric Bellm\tabularnewline
\bottomrule
\end{longtable}

\subsubsection{Test Items}\label{test-items-7}

This test will check that the full stream of LSST alerts can be
distributed to end users.\\[2\baselineskip]Specifically, this will
demonstrate that:

\begin{itemize}
\tightlist
\item
  Serialized alert packets can be loaded into the alert distribution
  system at LSST-relevant scales (10,000 alerts every 39 seconds);
\item
  Alert packets can be retrieved from the queue system at LSST-relevant
  scales.
\end{itemize}

\subsubsection{Requirements}\label{requirements-7}

\begin{itemize}
\tightlist
\item
  \href{https://jira.lsstcorp.org/browse/LVV-3}{LVV-3} -
  DMS-REQ-0002-V-01: Transient Alert Distribution
\end{itemize}

\subsubsection{Precondition}\label{precondition-6}

Input data: A sample of Avro-formatted alert packets.

\subsubsection{Test Script}\label{test-script-7}

\textbf{Step 1}\\
Download Kafka Docker image from
https://github.com/lsst-dm/alert\textbackslash{}\_stream.\\[2\baselineskip]\textbf{Step
2}\\
Change to the alert\_stream directory and build the docker image.\\

\begin{verbatim}
docker build -t "lsst-kub001:5000/alert_stream" .
\end{verbatim}

\textbf{Step 3}\\
From the alert\_stream/kubernetes directory, start Kafka and
Zookeeper:\\[2\baselineskip]

\begin{verbatim}
kubectl create -f zookeeper-service.yaml
kubectl create -f zookeeper-deployment.yaml
kubectl create -f kafka-deployment.yaml
kubectl create -f kafka-service.yaml
\end{verbatim}

(use kubectl get pods/services between each command to check status;
wait until each is ``Running'' before starting the next
command)\\[4\baselineskip]\textbf{Step 4}\\
Confirm Kafka and Zookeeper are listed when
running\\[2\baselineskip]kubectl get
pods\\[2\baselineskip]and\\[2\baselineskip]kubectl get
services\\[2\baselineskip]\textbf{Step 5}\\
Register it with Kubernetes\\[2\baselineskip]docker push
lsst-kub001:5000/alert\_stream\\[2\baselineskip]\textbf{Step 6}\\
Start a consumer that monitors the full stream and logs a deserialized
version of every Nth packet:\\

\begin{verbatim}
kubectl create -f consumerall-deployment.yaml
\end{verbatim}

\textbf{Step 7}\\

\begin{verbatim}
Start a producer that reads alert packets from disk and loads them into the Kafka queue:
\end{verbatim}

\begin{verbatim}
kubectl create -f sender-deployment.yaml
\end{verbatim}

\textbf{Step 8}\\
Determine the name of the alert sender pod with\\[2\baselineskip]kubectl
get pods\\[2\baselineskip]Examine output log
files.\\[2\baselineskip]kubectl logs \textless{}pod
name\textgreater{}\\[2\baselineskip]Verify that alerts are being sent
within 40 seconds by subtracting the timing
measurements.\\[2\baselineskip]\textbf{Step 9}\\
Determine the name of the consumer pod with\\[2\baselineskip]kubectl get
pods\\[2\baselineskip]Examine output log files.\\[2\baselineskip]kubectl
logs \textless{}pod name\textgreater{}\\[2\baselineskip]The packet log
should show deserialized alert packets with contents matching the input
packets.\\[4\baselineskip]

\hypertarget{lvv-t218---ad-00-10-simple-filtering-of-the-lsst-alert-stream}{\subsection{\texorpdfstring{\href{https://jira.lsstcorp.org/secure/Tests.jspa\#/testCase/LVV-T218}{LVV-T218}
- AD-00-10: Simple Filtering of the LSST Alert
Stream}{LVV-T218 - AD-00-10: Simple Filtering of the LSST Alert Stream}}\label{lvv-t218---ad-00-10-simple-filtering-of-the-lsst-alert-stream}}

\begin{longtable}[]{@{}llllll@{}}
\toprule
Version & Status & Priority & Verification Type & Critical Event &
Owner\tabularnewline
\midrule
\endhead
1 & Draft & Normal & Test & False & Eric Bellm\tabularnewline
\bottomrule
\end{longtable}

\subsubsection{Test Items}\label{test-items-8}

This test will demonstrate the ``mini-broker'' filtering service that
returns a subset of alerts from the full stream identified by
user-provided filters.\\[2\baselineskip]Specifically, this will
demonstrate that:\\

\begin{itemize}
\tightlist
\item
  The filtering service can retrieve alerts from the full alert stream
  and filter them according to their contents; ~ ~
\item
  The filtered subset can be delivered to science users.
\end{itemize}

\subsubsection{Requirements}\label{requirements-8}

\begin{itemize}
\tightlist
\item
  \href{https://jira.lsstcorp.org/browse/LVV-173}{LVV-173} -
  DMS-REQ-0342-V-01: Alert Filtering Service
\item
  \href{https://jira.lsstcorp.org/browse/LVV-179}{LVV-179} -
  DMS-REQ-0348-V-01: Pre-defined alert filters
\item
  \href{https://jira.lsstcorp.org/browse/LVV-174}{LVV-174} -
  DMS-REQ-0343-V-01: Performance Requirements for LSST Alert Filtering
  Service
\end{itemize}

\subsubsection{Precondition}\label{precondition-7}

Input data: A sample of Avro-formatted alert packets derived from LSST
simulations corresponding to one night of simulated LSST observing.

\subsubsection{Test Script}\label{test-script-8}

\textbf{Step 1}\\
Download Kafka Docker image from
https://github.com/lsst-dm/alert\textbackslash{}\_stream.\\[2\baselineskip]\textbf{Step
2}\\
Change to the alert\_stream directory and build the docker image.\\

\begin{verbatim}
docker build -t "lsst-kub001:5000/alert_stream" .
\end{verbatim}

\textbf{Step 3}\\
From the alert\_stream/kubernetes directory, start Kafka and
Zookeeper:\\[2\baselineskip]

\begin{verbatim}
kubectl create -f zookeeper-service.yaml
kubectl create -f zookeeper-deployment.yaml
kubectl create -f kafka-deployment.yaml
kubectl create -f kafka-service.yaml
\end{verbatim}

(use kubectl get pods/services between each command to check status;
wait until each is ``Running'' before starting the next
command)\\[4\baselineskip]\textbf{Step 4}\\
Confirm Kafka and Zookeeper are listed when
running\\[2\baselineskip]kubectl get
pods\\[2\baselineskip]and\\[2\baselineskip]kubectl get
services\\[2\baselineskip]\textbf{Step 5}\\
Register it with Kubernetes\\[2\baselineskip]docker push
lsst-kub001:5000/alert\_stream\\[2\baselineskip]\textbf{Step
6}\\[2\baselineskip]Start 100 consumers that consume the filtered
streams and logs a deserialized version of every Nth
packet:\\[2\baselineskip]

\begin{verbatim}
kubectl create -f consumer1-deployment.yaml
kubectl create -f consumer2-deployment.yaml
kubectl create -f consumer3-deployment.yaml
kubectl create -f consumer4-deployment.yaml
kubectl create -f consumer5-deployment.yaml
kubectl create -f consumer6-deployment.yaml
kubectl create -f consumer7-deployment.yaml
kubectl create -f consumer8-deployment.yaml
kubectl create -f consumer9-deployment.yaml
kubectl create -f consumer10-deployment.yaml
\end{verbatim}

\textbf{Step 7}\\
Start 5 filter groups:\\

\begin{verbatim}
kubectl create -f filterer1-deployment.yaml
kubectl create -f filterer2-deployment.yaml
kubectl create -f filterer3-deployment.yaml
kubectl create -f filterer4-deployment.yaml
kubectl create -f filterer5-deployment.yaml
\end{verbatim}

\textbf{Step 8}\\
Start a producer that reads alert packets from disk and loads them into
the Kafka queue:\\[2\baselineskip]

\begin{verbatim}
kubectl create -f sender-deployment.yaml
\end{verbatim}

\textbf{Step 9}\\
Determine the name of the alert sender pod with\\[2\baselineskip]kubectl
get pods\\[2\baselineskip]Examine output log
files.\\[2\baselineskip]kubectl logs \textless{}pod
name\textgreater{}\\[2\baselineskip]Verify that alerts are being sent
within 40 seconds by subtracting the timing
measurements.\\[2\baselineskip]\textbf{Step 10}\\
Determine the name of the consumer pods with\\[2\baselineskip]kubectl
get pods\\[2\baselineskip]Examine output log
files.\\[2\baselineskip]kubectl logs \textless{}pod
name\textgreater{}\\[2\baselineskip]The packet log should show
deserialized alert packets with contents matching the input
packets.\\[2\baselineskip]

\section{Requirements Traceability}\label{requirements-traceability}

\begin{longtable}[]{p{13cm}p{3cm}}
\toprule
Requirements & Test Cases\tabularnewline
\midrule
\endhead
\href{https://jira.lsstcorp.org/browse/LVV-3}{LVV-3 - DMS-REQ-0002-V-01:
Transient Alert Distribution} &
\protect\hyperlink{lvv-t217---ad-00-05-full-stream-alert-distribution}{LVV-T217}\tabularnewline
\href{https://jira.lsstcorp.org/browse/LVV-7}{LVV-7 - DMS-REQ-0010-V-01:
Difference Exposures} &
\protect\hyperlink{lvv-t18---ag-00-05-alert-generation-produces-required-data-products}{LVV-T18},
\protect\hyperlink{lvv-t20---ag-00-15-scientific-verification-of-difference-images}{LVV-T20}\tabularnewline
\href{https://jira.lsstcorp.org/browse/LVV-12}{LVV-12 -
DMS-REQ-0029-V-01: Generate Photometric Zeropoint for Visit Image} &
\protect\hyperlink{lvv-t19---ag-00-10-scientific-verification-of-processed-visit-images}{LVV-T19}\tabularnewline
\href{https://jira.lsstcorp.org/browse/LVV-13}{LVV-13 -
DMS-REQ-0030-V-01: Generate WCS for Visit Images} &
\protect\hyperlink{lvv-t19---ag-00-10-scientific-verification-of-processed-visit-images}{LVV-T19}\tabularnewline
\href{https://jira.lsstcorp.org/browse/LVV-29}{LVV-29 -
DMS-REQ-0069-V-01: Processed Visit Images} &
\protect\hyperlink{lvv-t18---ag-00-05-alert-generation-produces-required-data-products}{LVV-T18},
\protect\hyperlink{lvv-t19---ag-00-10-scientific-verification-of-processed-visit-images}{LVV-T19}\tabularnewline
\href{https://jira.lsstcorp.org/browse/LVV-30}{LVV-30 -
DMS-REQ-0070-V-01: Generate PSF for Visit Images} &
\protect\hyperlink{lvv-t19---ag-00-10-scientific-verification-of-processed-visit-images}{LVV-T19}\tabularnewline
\href{https://jira.lsstcorp.org/browse/LVV-31}{LVV-31 -
DMS-REQ-0072-V-01: Processed Visit Image Content} &
\protect\hyperlink{lvv-t19---ag-00-10-scientific-verification-of-processed-visit-images}{LVV-T19}\tabularnewline
\href{https://jira.lsstcorp.org/browse/LVV-32}{LVV-32 -
DMS-REQ-0074-V-01: Difference Exposure Attributes} &
\protect\hyperlink{lvv-t20---ag-00-15-scientific-verification-of-difference-images}{LVV-T20}\tabularnewline
\href{https://jira.lsstcorp.org/browse/LVV-100}{LVV-100 -
DMS-REQ-0269-V-01: DIASource Catalog} &
\protect\hyperlink{lvv-t18---ag-00-05-alert-generation-produces-required-data-products}{LVV-T18},
\protect\hyperlink{lvv-t21---ag-00-20-scientific-verification-of-diasource-catalog}{LVV-T21}\tabularnewline
\href{https://jira.lsstcorp.org/browse/LVV-101}{LVV-101 -
DMS-REQ-0270-V-01: Faint DIASource Measurements} &
\protect\hyperlink{lvv-t21---ag-00-20-scientific-verification-of-diasource-catalog}{LVV-T21}\tabularnewline
\href{https://jira.lsstcorp.org/browse/LVV-102}{LVV-102 -
DMS-REQ-0271-V-01: DIAObject Catalog} &
\protect\hyperlink{lvv-t18---ag-00-05-alert-generation-produces-required-data-products}{LVV-T18},
\protect\hyperlink{lvv-t22---ag-00-25-scientific-verification-of-diaobject-catalog}{LVV-T22}\tabularnewline
\href{https://jira.lsstcorp.org/browse/LVV-103}{LVV-103 -
DMS-REQ-0272-V-01: DIAObject Attributes} &
\protect\hyperlink{lvv-t22---ag-00-25-scientific-verification-of-diaobject-catalog}{LVV-T22}\tabularnewline
\href{https://jira.lsstcorp.org/browse/LVV-116}{LVV-116 -
DMS-REQ-0285-V-01: Level 1 Source Association} &
\protect\hyperlink{lvv-t22---ag-00-25-scientific-verification-of-diaobject-catalog}{LVV-T22}\tabularnewline
\href{https://jira.lsstcorp.org/browse/LVV-139}{LVV-139 -
DMS-REQ-0308-V-01: Software Architecture to Enable Community Re-Use} &
\protect\hyperlink{lvv-t17---ag-00-00-installation-of-the-alert-generation-science-payload}{LVV-T17},
\protect\hyperlink{lvv-t216---ad-00-00-installation-of-the-alert-distribution-payloads}{LVV-T216}\tabularnewline
\href{https://jira.lsstcorp.org/browse/LVV-158}{LVV-158 -
DMS-REQ-0327-V-01: Background Model Calculation} &
\protect\hyperlink{lvv-t19---ag-00-10-scientific-verification-of-processed-visit-images}{LVV-T19}\tabularnewline
\href{https://jira.lsstcorp.org/browse/LVV-162}{LVV-162 -
DMS-REQ-0331-V-01: Computing Derived Quantities} &
\protect\hyperlink{lvv-t21---ag-00-20-scientific-verification-of-diasource-catalog}{LVV-T21},
\protect\hyperlink{lvv-t22---ag-00-25-scientific-verification-of-diaobject-catalog}{LVV-T22}\tabularnewline
\href{https://jira.lsstcorp.org/browse/LVV-173}{LVV-173 -
DMS-REQ-0342-V-01: Alert Filtering Service} &
\protect\hyperlink{lvv-t218---ad-00-10-simple-filtering-of-the-lsst-alert-stream}{LVV-T218}\tabularnewline
\href{https://jira.lsstcorp.org/browse/LVV-174}{LVV-174 -
DMS-REQ-0343-V-01: Performance Requirements for LSST Alert Filtering
Service} &
\protect\hyperlink{lvv-t218---ad-00-10-simple-filtering-of-the-lsst-alert-stream}{LVV-T218}\tabularnewline
\href{https://jira.lsstcorp.org/browse/LVV-178}{LVV-178 -
DMS-REQ-0347-V-01: Measurements in catalogs} &
\protect\hyperlink{lvv-t21---ag-00-20-scientific-verification-of-diasource-catalog}{LVV-T21},
\protect\hyperlink{lvv-t22---ag-00-25-scientific-verification-of-diaobject-catalog}{LVV-T22}\tabularnewline
\href{https://jira.lsstcorp.org/browse/LVV-179}{LVV-179 -
DMS-REQ-0348-V-01: Pre-defined alert filters} &
\protect\hyperlink{lvv-t218---ad-00-10-simple-filtering-of-the-lsst-alert-stream}{LVV-T218}\tabularnewline
\bottomrule
\end{longtable}




\section{The DECam ```HiTS'' dataset}

We use a subset of the DECam hits dataset, contained in the repository
\url{https://github.com/lsst/ap\_verify\_hits2015.git}.  As described in
\url{https://dmtn-039.lsst.io/}, we select HiTS fields \texttt{Blind15A\_26},
\texttt{Blind15A\_40}, and \texttt{Blind15A\_42}. We construct templates from the best-seeing observations of
same region of sky using the previous year's observations, labelled
\texttt{Blind14A\_04}, \texttt{Blind14A\_10}, and \texttt{Blind14A\_09}.

The specific visits we use are:

410915, 410929, 410931, 410971, 410985, 410987, 411021, 411035,
        411037, 411055, 411069, 411071, 411255, 411269, 411271 , 411305,
        411319, 411321, 411355, 411369, 411371, 411406, 411420, 411422 ,
        411456, 411470, 411472, 411657, 411671, 411673, 411707, 411721, 411724,
        411758, 411772, 411774, 411808, 411822, 411824, 411858, 411872,
        411874 , 412060, 412074, 412076, 412250, 412264, 412266, 412307,
        412321, 412324, 412504, 412518, 412520, 412554, 412568, 412570, 412604,
        412618, 412620, 412654, 412668, 412670, 412704, 412718, 412720, 413635,
        413649, 413651, 413680, 413694, 413696, 415314, 415328, 415330, 415364,
        415378, 415380, 419791, 419802, 419804, 421590, 421604, 421606.

For each visit we exclude CCDs 1, 2, and 61, leaving CCDs 3-60 and 62.  We use $g$-band only for these tests due to the need to build templates.


\end{document}
