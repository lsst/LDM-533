\documentclass[DM,lsstdraft,STS,toc]{lsstdoc}
\usepackage{enumitem}
\input meta.tex

\begin{document}

\def\product{LSST Level 1 System}

\setDocCompact{true}

\title[Test Spec for \product]{\product~Test Specification}

\author{Eric C. Bellm}
\setDocRef{\lsstDocType-\lsstDocNum}
\setDocDate{\vcsdate}

\setDocAbstract {
This document describes the detailed test specification for the \product{}.
}

% Most recent last
\setDocChangeRecord{%
	\addtohist{1}{2017-11-XX}{Initial release of draft version.}{Bellm}
}

\setDocCurator{Eric C. Bellm}
\setDocUpstreamLocation{\url{https://github.com/lsst/ldm-533}}
\setDocUpstreamVersion{\vcsrevision}

\maketitle

\section{Introduction}
\label{sec:intro}

This document specifies the test procedure for the \product{}.

The \product{} is the component of the LSST system which is responsible for
scientific processing leading to:

\begin{itemize}

  \item{Single frame processing and measurement;}
  \item{Alert generation from difference image analysis;}
  \item{Alert distribution to community brokers;}
  \item{Simple filtering of alerts;}
  \item{Precovery and forced photometry measurements on new and previously-known
	  sources found in difference imaging;}
  \item{Identification of moving objects.}
  \item{Generating QC metrics based on pipeline execution and post-processing of
  scientific data products.}


\end{itemize}

A full description of this product is provided in \S6 (which describes the
Data Facility-provided execution services) and \S13.1 (the science payloads) of
\citeds{LDM-148}.

\subsection{Objectives}
\label{sec:objectives}

This document builds on the description of LSST Data Management's approach to
testing as described in \citeds{LDM-503} to describe the detailed tests that
will be performed on the \product{} as part of the verification of the DM system.

It identifies test designs, test cases and procedures for the tests, and the
pass/fail criteria for each test.

\subsection{Scope}
\label{sec:scope}

This document describes the test procedures for the following components of
the LSST system (as described in \citeds{LDM-148}):

\begin{itemize}

  \item{Services provided by the LSST Data Facility:

    \begin{itemize}
      \item{Prompt Processing Execution}
      \item{Batch and Offline Processing Execution}
      \item{Level 1 Quality Control}
      \item{Alert Distribution Execution}
      \item{Alert Filtering Execution}
    \end{itemize}
  }

  \item{Science payloads:

    \begin{itemize}
      \item{Single frame processing Payload}
      \item{Alert Generation Payload}
      \item{Precovery and Forced Photometry Payload}
      \item{MOPS Payload}
    \end{itemize}

  }

\end{itemize}

\subsection{Applicable Documents}
\label{sec:docs}

\addtocounter{table}{-1}

\begin{tabular}[htb]{l l}
\citeds{LDM-148} & LSST DM System Architecture \\
\citeds{LDM-151} & LSST DM Science Pipelines Design \\
\citeds{LDM-294} & LSST DM Organization \& Management \\
\citeds{LDM-502} & The Measurement and Verification of DM Key Performance Metrics \\
\citeds{LDM-503} & LSST DM Test Plan \\
\citeds{LSE-61}  & LSST DM Subsystem Requirements \\
\citeds{LSE-163} & LSST Data Products Definition Document \\
\end{tabular}

\subsection{References\label{sect:references}}
\renewcommand{\refname}{}
\bibliography{lsst,refs,books,refs_ads}

%\subsection{Definitions, acronyms, and abbreviations \label{sect:acronyms}} % include acronyms.tex generated by the acronyms.csh (GaiaTools)
%\input{acronyms}


%----------------------------------------------------
% TASK IDENTIFICATION - APPROACH
%----------------------------------------------------
\section{Approach}
\label{sec:approach}

The major activities to be performed are to:

\begin{itemize}

  \item{Compare the design of the Alert Production payload as
  implemented to the requirements on the outputs of the DM Subsystem as
  defined in \citeds{LSE-63} and \citeds{LSE-163} to demonstrate that all data
  products required by the scientific community will be delivered by the
  system as built.}

  \item{Ensure that all data products included in the AP payload design are
  correctly produced and persisted appropriately to the LSST 
  Data Backbone, Alert Distribution System, and/or Alert Filtering service
  as appropriate.}

  \item{Ensure that all data products required by the Precovery and Forced
  Photometry payload are correctly
  produced and persisted appropriately to the LSST Data Backbone.}

  \item{Ensure that all data products required by the MOPS system are correctly
  produced and persisted appropriately to the LSST Data Backbone.}

  \item{Demonstrate that QC metrics are properly calculated and transmitted
  during the execution all L1 production types.}

  \item{Demonstrate that post-processing QC analysis of data products can be
  used to identify and report on failures or anomalies in the processing.}

\end{itemize}

\subsection{Tasks and criteria}
\label{sec:tasks}

The following are the major items under test:

\begin{itemize}

  \item{The science payload capable of prompt processing of single visit
	  images;}

  \item{The Alert Generation payload that detects variable sources through
	  difference image analysis;}

  \item{The Alert Distribution System that packages alerts and forwards
	  them to community brokers;}

  \item{The filtering system that allows science users to apply simple
	  filters to the alert stream;} 


  \item{The Precovery and Forced Photometry payloads that measure flux
	  levels for new and previously-known sources found in difference
	  images;}
	
  \item{The Moving Object Processing System payload that identifies solar
	  system bodies from difference image sources;}

  \item{Services capable of scheduling and managing the execution of all of
  the above payloads, marshalling their results, and making them available to
  other parts of the system for analysis or further distribution.}

\end{itemize}

\subsection{Features to be tested}
\label{sec:feat2test}

\begin{itemize}

  \item{Execution of payloads described in \S\ref{sec:tasks};}
  \item{Persistence of all required data products;}
  \item{Scientific fidelity of those data products: do they satisfy the
  requirements described in \citeds{LSE-61}?}

\end{itemize}

\subsection{Features not to be tested}
\label{sec:featnot2test}

This document does not describe facilities for periodically generating or
collecting key performance metrics (KPMs), except insofar as those KPMs are
incidentally measured as part of executing the documented testcases. The KPMs
and the system being used to track KPMs and to ensure compliance with
documented requirements is described in \citeds{LDM-502}.

\subsection{Pass/fail criteria}
\label{sec:passfail}

The results of all tests will be assessed using the criteria described in
\citeds{LDM-503} \S4.

Note that, when executing pipelines, tasks or individual algorithms, any
unexplained or unexpected errors or warnings appearing in the associated log
or on screen output must be described in the documentation for the system
under test. Any warning or error for which this is not the case must be filed
as a software problem report and filed with the DMCCB.

\subsection{Suspension criteria and resumption requirements}
\label{suspension}

Refer to individual test cases where applicable.

\subsection{Naming convention}

All tests are named according to the pattern \textsc{prod-xx-yy} where:

\begin{description}[font=\normalfont\scshape]

  \item[prod]{The product under test. Relevant entries for this document are:
    \begin{description}[font=\normalfont\scshape,topsep=-1.0ex]
      \item[AG]{The Alert Generation payload and associated service}
      \item[AD]{The Alert Distribution payload and associated service}
      \item[AF]{The Alert Filtering service}
      \item[PFP]{The Precovery and Forced Photometry payload and associated
	      service}
      \item[MOPS]{The MOPS payload and associated service}
    \end{description}
  }
  \item[xx]{Test specification number (in increments of 10)}
  \item[yy]{Test case number (in increments of 5)}

\end{description}

%\section{Test Specification Design}

\subsection{AG-00: Small Scale Alert Generation Processing}
\label{ag-00}

\subsubsection{Objective}

This test specification demonstrates the successful execution of an
Alert Generation payload on a relatively small scale based on data from
precursor surveys.

It will demonstrate that:

\begin{itemize}

  \item{Science payload code can be made available on systems managed by the
  LSST Data Facility;}

  \item{The science payload can be executed under the
  control of the Offline Processing Execution service;}

  \item{All required science data products can be collected by the execution
  service and made available for subsequent analysis;}

  \item{The Alert Generation payload generates results broadly
  equivalent to ``native'' reductions of precursor survey data.}

\end{itemize}

Note that this test specification does not extend to demonstrating the
detailed compliance of LSST data products with all \citeds[Science
Requirements Document]{LPM-17} level requirements: such a demonstration would
require carefully curated LSST-like datasets (or simulated data), a detailed
understanding of the LSST system, LSST-like calibration products, etc, which
are assumed not to be available for this test.

\subsubsection{Approach refinements}

The general approach defined in \citeds{LDM-503} is used.

\subsubsection{Test case identification}

\begin{longtable} {|p{0.4\textwidth}|p{0.6\textwidth}|}\hline
\textbf{Test Case}  & \textbf{Description} \\\hline

\hyperref[ag-00-00]{AG-00-00} & Tests that the Alert Generation science payload can be installed on LSST Data Facility-managed systems.\\\hline
\hyperref[ag-00-05]{AG-00-05} & Tests that required data products are
produced by executing the Alert Generation payload. \\\hline
\hyperref[ag-00-10]{AG-00-10} & Tests that the delivered processed visit images meet scientific requirements. \\\hline
\hyperref[ag-00-15]{AG-00-15} & Tests that the delivered difference images meet scientific requirements. \\\hline
\hyperref[ag-00-20]{AG-00-20} & Tests that the delivered DIASource catalogs meet scientific requirements. \\\hline
\hyperref[ag-00-25]{AG-00-25} & Tests that the delivered DIAObject
	catalog meets scientific requirements. \\\hline
\end{longtable}


%\section{Test Case Specification}

\subsection{AG-00-00: Installation of the Alert Generation science payload.}
\label{ag-00-00}

\subsubsection{Requirements}

DMS-REQ-0308.

\subsubsection{Test items}

This test will check:

\begin{itemize}

  \item{That the Alert Generation science payload is available for
  distribution from documented channels;}

  \item{That the Alert Generation science payload can be installed on
  LSST Data Facility-managed systems.}

\end{itemize}

\subsubsection{Intercase dependencies}

None.

\subsubsection{Environmental needs}

\paragraph{Hardware}

This test case shall be executed on a developer system at NCSA which serves as
the ``head node'' or otherwise provides access to filesystems shared by the
LSST Verification Cluster (LSST-VC). We assume that this system will be
\texttt{lsst-dev01.ncsa.illinois.edu} and the filesystem will be a GPFS-based
system mounted at \texttt{/software}.

The test also requires access to one LSST-VC compute node.

\paragraph{Software}

All prerequisite packages listed at
\url{https://pipelines.lsst.io/install/prereqs/centos.html} must be available
on the test system and on the LSST-VC compute node.

\subsubsection{Input specification}

No input data is required for this test case.

\subsubsection{Output specification}

The Alert Generation science payload will be made available on a shared
filesystem accessible from LSST-VC compute notes.

\subsubsection{Procedure}

\begin{enumerate}

  \item{Release 14.0 of the LSST Science Pipelines will be installed into the
  GPFS filesystem accessible at \texttt{/software} on \texttt{lsst-dev01}
  following the instructions at
  \url{https://pipelines.lsst.io/install/newinstall.html}.}

  \item{The lsst\_distrib top level package will be enabled:

  \begin{verbatim}
  source /software/lsstsw/stack3/loadLSST.bash
  setup lsst_distrib
  \end{verbatim}
  }

  \item{The ``LSST Stack Demo'' package will be downloaded onto the test
  system from
  \url{https://github.com/lsst/lsst_dm_stack_demo/releases/tag/14.0} and
  uncompressed.}

  \item{The demo package will be executed by following the instructions in its
  ``README`` file. The string ``Ok.`` should be returned.}

  \item{A shell on an LSST-VC compute node will now be obtained by executing:

  \begin{verbatim}
  $ srun -I --pty bash
  \end{verbatim}
  }

  \item{The demo package will be executed on the compute node and the same
  result obtained.}

\end{enumerate}

\subsection{AG-00-05: Alert Generation Produces Required Data Products}
\label{ag-00-05}

\subsubsection{Requirements}

DMS-REQ-0069, DMS-REQ-0010, DMS-REQ-0269, DMS-REQ-0271

\subsubsection{Test items}
\label{ag-00-05-items}

This test will check that the basic data products produced by Alert
Generation are generated by execution of the science payload.

These products will include:

\begin{itemize}

  \item{Processed visit images (PVIs; DMS-REQ-0069);}
  \item{Difference Exposures (DMS-REQ-0010);}
  \item{DIASource catalogs (DMS-REQ-0269);}
  \item{DIAObject catalogs (DMS-REQ-0271);}

\end{itemize}

\subsubsection{Intercase dependencies}

\begin{itemize}

  \item{\hyperref[ag-00-00]{AG-00-00}}

\end{itemize}

\subsubsection{Environmental needs}

\paragraph{Hardware}

The test shall be carried out on a machine with at least 16\,GB of RAM and
multiple CPU cores which has access to the \texttt{/datasets} shared (GPFS)
filesystem at the LSST Data Facility.

\paragraph{Software}

Release 14.0 of the DM Software Stack will be pre-installed (following the
procedure described in \hyperref[ag-00-00]{AG-00-00}).

\subsubsection{Input specification}

A complete processing of the DECam ``HiTS'' dataset, as defined at
\url{https://dmtn-039.lsst.io/} and
\url{https://github.com/lsst/ap_verify_hits2015}, through the Alert
Generation science payload.

This dataset shall be made available in a standard LSST data repository,
accessible via the ``Data Butler''.

It is not required that all combinations of visit and CCD have been processed
successfully: a number of failures are expected. However, documentation to
describe processing failures should be provided.

\subsubsection{Output specification}

None.

\subsubsection{Procedure}

\begin{itemize}

  \item{The DM Stack and Alert Processing packaged shall be initialized 
  as described in \hyperref[ag-00-00]{AG-00-00}.}

  \item{The alert generation processing will be executed using the verification cluster:

  \begin{verbatim}
  python ap_verify/bin/prepare_demo_slurm_files.py

  # At present we must run a single ccd+visit to handle ingestion before 
  # parallel processing can begin
  ./ap_verify/bin/exec_demo_run_1ccd.sh 410915 25

  ln -s ap_verify/bin/demo_run.sl
  ln -s ap_verify/bin/demo_cmds.conf
  sbatch demo_run.sl
  \end{verbatim}

  and any errors or failures reported.}


  \item{A ``Data Butler'' will be initialized to access the repository.}

  \item{For each of the expected data products types (listed in \S\ref{ag-00-05-items})
  and each of the expected units (PVIs, catalogs, etc.), the data product will be
    retrieved from the Butler and verified to be non-empty.}

  \item{\texttt{DIAObjects} are currently only stored in a database, without
  shims to the Butler, so the existence of the database table and its
  non-empty contents will be verified by directly accessing it using
  \texttt{sqlite3} and executing appropriate SQL queries.}

\end{itemize}

\subsection{AG-00-10: Scientific Verification of Processed Visit Images}
\label{ag-00-10}

\subsubsection{Requirements}

DMS-REQ-0069, DMS-REQ-0327, DMS-REQ-0029, DMS-REQ-0070,
DMS-REQ-0030, DMS-REQ-0072.

\subsubsection{Test items}
\label{ag-00-10-items}

This test will check that the Processed Visit Images (PVIs) delivered by the
alert generation science payload meet the requirements laid down by \citeds{LSE-61}.

Specifically, this will demonstrate that:

\begin{itemize}

  \item{Processed visit images have been generated and persisted during
  payload execution;}
  \item{Each PVI includes a science pixel array, a mask array, and a variance array. (DMS-REQ-0072).}
  \item{Each PVI includes a background model (DMS-REQ-0327), photometric
  zero-point (DMS-REQ-0029), spatially-varying PSF (DMS-REQ-0070) and WCS
  (DMS-REQ-0030).}
  \item{Saturated pixels are correctly masked.}
  \item{Pixels affected by cosmic rays are correctly masked.}
  \item{The background is not oversubtracted around bright objects.}

\end{itemize}

This test does not include quantitative targets for the science quality criteria.

\subsubsection{Intercase dependencies}

\begin{itemize}

  \item{\hyperref[ag-00-00]{AG-00-00}}
  \item{\hyperref[ag-00-05]{AG-00-05}}

\end{itemize}

\subsubsection{Environmental needs}

\paragraph{Hardware}

The test shall be carried out on a machine with at least 16\,GB of RAM and
multiple CPU cores which has access to the \texttt{/datasets} shared (GPFS)
filesystem at the LSST Data Facility.

\paragraph{Software}

Release 14.0 of the DM Software Stack will be pre-installed (following the
procedure described in \hyperref[ag-00-00]{AG-00-00}).

\subsubsection{Input specification}

A complete processing of the DECam ``HiTS'' dataset, as defined at
\url{https://dmtn-039.lsst.io/} and
\url{https://github.com/lsst/ap_verify_hits2015}, through the Alert
Generation science payload.

This dataset shall be made available in a standard LSST data repository,
accessible via the ``Data Butler''.

It is not required that all combinations of visit and CCD have been processed
successfully: a number of failures are expected. However, documentation to
describe processing failures should be provided.

\subsubsection{Output specification}

None.

\subsubsection{Procedure}

\begin{itemize}

  \item{The DM Stack shall be initialized using the \texttt{loadLSST} script
  (as described in \hyperref[ag-00-00]{AG-00-00}).}

  \item{A ``Data Butler'' will be initialized to access the repository.}

  \item{For each processed CCD, the PVI will be retrieved from the Butler, and
  the existence of all components described in \S\ref{ag-00-10-items} will be
  verified.}

  \item{Five sensors will be chosen at random from each of two visits and inspected by eye for unmasked artifacts.}

\end{itemize}

\subsection{AG-00-15: Scientific Verification of Difference Images}
\label{ag-00-15}

\subsubsection{Requirements}

DMS-REQ-0010, DMS-REQ-0074, 

\subsubsection{Test items}
\label{ag-00-15-items}

This test will check that the difference images delivered by the
Alert Generation science payload meet the requirements laid down by \citeds{LSE-61}.

Specifically, this will demonstrate that:

\begin{itemize}

  \item{Difference images have been generated and persisted during
  payload execution;}
  \item{Each difference image includes information about the identity of
	the input exposures, and metadata such as a representation of the
		PSF matching kernel (DMS-REQ-0074);}
  \item{Masks are correctly propagated from the input images.}

\end{itemize}

This test does not include quantitative targets for the science quality criteria; we instead require for each test that we be able to quickly construct a plot or display summary images that allow such a target can be visualized.

\subsubsection{Intercase dependencies}

\begin{itemize}

  \item{\hyperref[ag-00-00]{AG-00-00}}
  \item{\hyperref[ag-00-05]{AG-00-05}}

\end{itemize}

\subsubsection{Environmental needs}

\paragraph{Hardware}

The test shall be carried out on a machine with at least 16\,GB of RAM and
multiple CPU cores which has access to the \texttt{/datasets} shared (GPFS)
filesystem at the LSST Data Facility.

\paragraph{Software}

Release 14.0 of the DM Software Stack will be pre-installed (following the
procedure described in \hyperref[ag-00-00]{AG-00-00}).

\subsubsection{Input specification}

A complete processing of the DECam ``HiTS'' dataset, as defined at
\url{https://dmtn-039.lsst.io/} and
\url{https://github.com/lsst/ap_verify_hits2015}, through the Alert
Generation science payload.

This dataset shall be made available in a standard LSST data repository,
accessible via the ``Data Butler''.

It is not required that all combinations of visit and CCD have been processed
successfully: a number of failures are expected. However, documentation to
describe processing failures should be provided.

\subsubsection{Output specification}

None.

\subsubsection{Procedure}

\begin{itemize}

  \item{The DM Stack shall be initialized using the \texttt{loadLSST} script
  (as described in \hyperref[ag-00-00]{AG-00-00}).}

  \item{A ``Data Butler'' will be initialized to access the repository.}

  \item{For each processed CCD, the difference image 
	  will be retrieved from the Butler, and
  the existence of all components described in \S\ref{ag-00-15-items} will be
  verified.}

  \item{Scripts from the \texttt{pipe\_analysis} package will be run on every visit to check for the presence of data products and make plots}

  \item{Five sensors will be chosen at random from each of two visits and inspected by eye for unmasked artifacts.}

\end{itemize}

\subsection{AG-00-20: Scientific Verification of DIASource Catalog}
\label{ag-00-20}

\subsubsection{Requirements}

DMS-REQ-0269, DMS-REQ-0270, DMS-REQ-0347, DMS-REQ-0331.

\subsubsection{Test items}
\label{ag-00-20-items}

This test will check that the difference image source catalogs
delivered by the Alert Generation science
payload meet the requirements laid down by \citeds{LSE-61}.

Specifically, this will demonstrate that:

\begin{itemize}

  \item{Measurements in the catalog are presented in flux units (DMS-REQ-0347);}

  \item{Each DIASource record contains an appropriate subset of the attributes
  required by DMS-REQ-0269. In particular, the LDM-503-3-era pipeline is
  expected to provide DIASource positions (sky and focal plane), fluxes, and
  flags indicative of issues encountered during processing.}

  \item{Faint DIASources satisfying additional criteria are stored (DMS-REQ-0270).}

  \item{Derived quantities are provided in pre-computed columns (DMS-REQ-0331);}

\end{itemize}

This test does not include quantitative targets for the science quality criteria.

\subsubsection{Intercase dependencies}

\begin{itemize}

  \item{\hyperref[ag-00-00]{AG-00-00}}
  \item{\hyperref[ag-00-05]{AG-00-05}}

\end{itemize}

\subsubsection{Environmental needs}

\paragraph{Hardware}

The test shall be carried out on a machine with at least 16\,GB of RAM and
multiple CPU cores which has access to the \texttt{/datasets} shared (GPFS)
filesystem at the LSST Data Facility.

\paragraph{Software}

Release 14.0 of the DM Software Stack will be pre-installed (following the
procedure described in \hyperref[ag-00-00]{AG-00-00}).

\subsubsection{Input specification}

A complete processing of the DECam ``HiTS'' dataset, as defined at
\url{https://dmtn-039.lsst.io/} and
\url{https://github.com/lsst/ap_verify_hits2015}, through the Alert
Generation science payload.

This dataset shall be made available in a standard LSST data repository,
accessible via the ``Data Butler''.

It is not required that all combinations of visit and CCD have been processed
successfully: a number of failures are expected. However, documentation to
describe processing failures should be provided.

\subsubsection{Output specification}

None.

\subsubsection{Procedure}

\begin{itemize}

  \item{The DM Stack shall be initialized using the \texttt{loadLSST} script
  (as described in \hyperref[ag-00-00]{AG-00-00}).}

  \item{A ``Data Butler'' will be initialized to access the repository.}

  \item{DIASource records will be accessed by querying the Butler, then
  examined interactively at a Python prompt.}

\end{itemize}

\subsection{AG-00-25: Scientific Verification of DIAObject Catalog}
\label{AG-00-25}

\subsubsection{Requirements}

DMS-REQ-0285, DMS-REQ-0271, DMS-REQ-0272, DMS-REQ-0347, DMS-REQ-0331

\subsubsection{Test items}
\label{ag-00-25-items}

This test will check that the DIAObject catalogs delivered by the Alert
Generation science
payload meet the requirements laid down by \citeds{LSE-61}.

Specifically, this will demonstrate that:

\begin{itemize}

  \item{Measurements in the catalog are presented in flux units
  (DMS-REQ-0347);}
\item{Each DIAObject record contains the required attributes (DMS-REQ-2071 \&
	DMS-REQ-0130);}
\item{Each DIAObject record characterizes variability using
	DIASources contained in the required time interval
	(DMS-REQ-0319).}
 \item{Relevant derived quantities are provided in pre-computed columns
  (DMS-REQ-0331);}
\end{itemize}

This test does not include quantitative targets for the science quality criteria; we instead require for each test that we be able to quickly construct a plot in which such a target can be visualized.

All science quality tests in this section shall distinguish between blended and isolated objects.

\subsubsection{Intercase dependencies}

\begin{itemize}

  \item{\hyperref[ag-00-00]{AG-00-00}}
  \item{\hyperref[ag-00-05]{AG-00-05}}

\end{itemize}

\subsubsection{Environmental needs}

\paragraph{Hardware}

The test shall be carried out on a machine with at least 16\,GB of RAM and
multiple CPU cores which has access to the \texttt{/datasets} shared (GPFS)
filesystem at the LSST Data Facility.

\paragraph{Software}

Release 14.0 of the DM Software Stack will be pre-installed (following the
procedure described in \hyperref[ag-00-00]{AG-00-00}).

\subsubsection{Input specification}

A complete processing of the DECam ``HiTS'' dataset, as defined at
\url{https://dmtn-039.lsst.io/} and
\url{https://github.com/lsst/ap_verify_hits2015}, through the Alert
Generation science payload.

This dataset shall be made available in a standard LSST data repository,
accessible via the ``Data Butler''.

It is not required that all combinations of visit and CCD have been processed
successfully: a number of failures are expected. However, documentation to
describe processing failures should be provided.

\subsubsection{Output specification}

None.

\subsubsection{Procedure}

\begin{itemize}

  \item{The DM Stack shall be initialized using the \texttt{loadLSST} script
  (as described in \hyperref[ag-00-00]{AG-00-00}).}

  \item{\texttt{sqlite3} or python's \texttt{sqlalchemy} module will be used 
	  to access the Level 1 database.}

\end{itemize}


\subsection{AD-00-00: Installation of the Alert Distribution payloads.}
\label{ad-00-00}

\subsubsection{Requirements}

DMS-REQ-0308.

\subsubsection{Test items}

This test will check:

\begin{itemize}

  \item{That the Alert Distribution payloads are available 
  from documented channels;}

  \item{That the Alert Distribution payloads can be installed on
  LSST Data Facility-managed systems.}

  \item{That the Alert Distribution payloads can be executed by
  LSST Data Facility-managed systems.}

\end{itemize}

\subsubsection{Intercase dependencies}

None.

\subsubsection{Environmental needs}

\paragraph{Hardware}

This test case shall be executed on the \textit{Kubernetes Commons} at the LDF. 
As discussed in https://dmtn-028.lsst.io/ and https://dmtn-081.lsst.io/, the test machine should have at least 16 cores, 64 GB of memory and access to at least 1.5 TB of shared storage.

\paragraph{Software}

No software outside of the components to be installed in this case are required.

\subsubsection{Input specification}

No input data is required for this test case.

\subsubsection{Output specification}

The Alert Distribution payloads will be executed by the 
filesystem accessible from LSST-VC compute notes.

\subsubsection{Procedure}

\begin{enumerate}


	\item{Download Kafka Docker image from \url{https://github.com/lsst-dm/alert\_stream}.}
\textbf{This Docker Swarm setup needs to be updated for Kubernetes:}
	\item{Build the docker containers: \texttt{\$ docker build -t "alert\_stream" .}}
	\item{From the \texttt{alert\_stream} directory, bring up the container networking and start Kafka and Zookeeper:

\begin{verbatim}
docker network create --driver overlay alert_stream_default

docker service create \
        --name kafka \
        --network alert_stream_default \
        --constraint node.role==manager \
        -p 9092 \
        -e KAFKA_BROKER_ID=1 \
        -e KAFKA_ZOOKEEPER_CONNECT=zookeeper:32181 \
        -e KAFKA_ADVERTISED_LISTENERS=PLAINTEXT://kafka:9092 \
        -e KAFKA_OFFSETS_TOPIC_REPLICATION_FACTOR=1 \ # remove if starting 3 brokers or more
        -e KAFKA_HEAP_OPTS="-Xmx8g -Xms8g" \
        -e KAFKA_JVM_PERFORMANCE_OPTS="-XX:MetaspaceSize=96m -XX:+UseG1GC -XX:MaxGCPauseMillis=20 -XX:InitiatingHeapOccupancyPercent=35 -XX:G1HeapRegionSize=16M -XX:MinMetaspaceFreeRatio=50 -XX:MaxMetaspaceFreeRatio=80" \
        confluentinc/cp-kafka:4.1.1

docker service create \
        --name zookeeper \
        --network alert_stream_default \
        --constraint node.role==manager \
        -p 32181 \
        -e ZOOKEEPER_CLIENT_PORT=32181 \
        -e ZOOKEEPER_TICK_TIME=2000 \
        confluentinc/cp-zookeeper:4.1.1
\end{verbatim}}

	\item{Confirm Kafka and Zookeeper are listed when running \texttt{docker service ls}.}

	\item{Confirm that the payload can execute: \texttt{\$ docker run -it alert\_stream python bin/sendAlertStream.py -h}.  A help string should be printed to the terminal.}

\end{enumerate}

\subsection{AD-00-05: Full Stream Alert Distribution}
\label{ad-00-05}

\subsubsection{Requirements}

DMS-REQ-0002

\subsubsection{Test items}
\label{ad-00-05-items}

This test will check that the full stream of LSST alerts can be distributed to end users.

Specifically, this will demonstrate that:

\begin{itemize}

  \item{Serialized alert packets can be loaded into the alert distribution system at LSST-relevant scales;}
  \item{Alert packets can be retrieved from the queue system at LSST-relevant scales.}

\end{itemize}

This test does not include quantitative targets for latency.

\subsubsection{Intercase dependencies}

\begin{itemize}

  \item{\hyperref[ad-00-00]{AD-00-00}}

\end{itemize}

\subsubsection{Environmental needs}

\paragraph{Hardware}

The test shall be carried out on a machine with at least 16\,GB of RAM and
multiple CPU cores which has access to the \texttt{/datasets} shared (GPFS)
filesystem at the LSST Data Facility.

\paragraph{Software}

The Kafka cluster and Zookeeper shall be instantiated according to the 
procedure described in \hyperref[ad-00-00]{AD-00-00}).

\subsubsection{Input specification}

A sample of Avro-formatted alert packets derived from 

This dataset shall be made available \textit{how?  directory of alerts on datasets?}

\subsubsection{Output specification}

None.

\subsubsection{Procedure}

\begin{itemize}

  \item{instructions for loading packets into running Kafka system}

  \item{instructions for retrieving packets}

  \item{instructions for monitoring what is retreived}

\end{itemize}

\subsection{AD-00-10: Simple Filtering of the LSST Alert Stream}
\label{ad-00-10}

\subsubsection{Requirements}

DMS-REQ-0342, DMS-REQ-0348, DMS-REQ-0343

\subsubsection{Test items}
\label{ag-00-10-items}

This test will demonstrate the ``mini-broker'' filtering service that 

Specifically, this will demonstrate that:

\begin{itemize}

	\item{The filtering service can retrieve alerts from the full alert stream and filter them according to their contents;}
	\item{The filtered subset can be delivered to science users.}

\end{itemize}

This test does not include quantitative targets for latency.

\subsubsection{Intercase dependencies}

\begin{itemize}

  \item{\hyperref[ad-00-00]{AD-00-00}}
  \item{\hyperref[ad-00-05]{AD-00-05}}

\end{itemize}

\subsubsection{Environmental needs}

\paragraph{Hardware}

The test shall be carried out on a machine with at least 16\,GB of RAM and
multiple CPU cores which has access to the \texttt{/datasets} shared (GPFS)
filesystem at the LSST Data Facility.

\paragraph{Software}

The Kafka cluster and Zookeeper shall be instantiated according to the 
procedure described in \hyperref[ad-00-00]{AD-00-00}).

\subsubsection{Input specification}

A sample of Avro-formatted alert packets derived from 

This dataset shall be made available \textit{how?  directory of alerts on datasets?}

\subsubsection{Output specification}

None.

\subsubsection{Procedure}

\begin{itemize}

  \item{instructions for making the mini-broker system listen to the Kafka system}

  \item{instructions for configuring filters}

  \item{instructions for loading packets into running Kafka system}

  \item{instructions for redirecting output to new Kafka topics?}

  \item{instructions for retrieving packets}

  \item{instructions for monitoring what is retreived}

\end{itemize}

\input{cases/ad-00-15.tex}


\end{document}
