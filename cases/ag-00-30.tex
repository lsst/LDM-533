\subsection{AG-00-30: Scientific Verification of DIAObject Catalog}
\label{AG-00-30}

\subsubsection{Requirements}

DMS-REQ-0285, DMS-REQ-0271, DMS-REQ-0272, DMS-REQ-0347, DMS-REQ-0331

\subsubsection{Test items}
\label{ag-00-30-items}

This test will check that the DIAObject catalogs delivered by the Alert
Generation science
payload meet the requirements laid down by \citeds{LSE-61}.

Specifically, this will demonstrate that:

\begin{itemize}

  \item{Measurements in the catalog are presented in flux units
  (DMS-REQ-0347);}
  \item{Derived quantities are provided in pre-computed columns
  (DMS-REQ-0331);}
\item{Each DIAObject record contains the required attributes (DMS-REQ-2071 \&
	DMS-REQ-0130);}
	\item{Each DIAObject record characterizes variability using
		DIASources contained in the required time interval
		(DMS-REQ-0319).}
\end{itemize}

This test does not include quantitative targets for the science quality criteria; we instead require for each test that we be able to quickly construct a plot in which such a target can be visualized.

All science quality tests in this section shall distinguish between blended and isolated objects.

\subsubsection{Intercase dependencies}

\begin{itemize}

  \item{\hyperref[ag-00-00]{AG-00-00}}
  \item{\hyperref[ag-00-10]{AG-00-10}}

\end{itemize}

\subsubsection{Environmental needs}

\paragraph{Hardware}

The test shall be carried out on a machine with at least 16\,GB of RAM and
multiple CPU cores which has access to the \texttt{/datasets} shared (GPFS)
filesystem at the LSST Data Facility.

\paragraph{Software}

Release 14.0 of the DM Software Stack will be pre-installed (following the
procedure described in \hyperref[drp-00-00]{DRP-00-00}).

\subsubsection{Input specification}

A complete processing of the DECam ``HiTS'' dataset, as defined at
\url{https://dmtn-039.lsst.io/} and
\url{https://github.com/lsst/ap_verify_hits2015}, through the Alert
Generation science payload.

This dataset shall be made available in a standard LSST data repository,
accessible via the ``Data Butler''.

It is not required that all combinations of visit and CCD have been processed
successfully: a number of failures are expected. However, documentation to
describe processing failures should be provided.

\subsubsection{Output specification}

None.

\subsubsection{Procedure}

\begin{itemize}

  \item{The DM Stack shall be initialized using the \texttt{loadLSST} script
  (as described in \hyperref[ag-00-00]{AG-00-00}).}

  \item{A ``Data Butler'' will be initialized to access the repository.}

  \item{Scripts from the \texttt{pipe\_analysis} package will be run on
	  every tract to check for the presence of data products and make
		plots.}

\end{itemize}
