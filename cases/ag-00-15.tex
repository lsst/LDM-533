\subsection{AG-00-15: Scientific Verification of Difference Images}
\label{ag-00-15}

\subsubsection{Requirements}

DMS-REQ-0010, DMS-REQ-0074, 

\subsubsection{Test items}
\label{ag-00-15-items}

This test will check that the difference images delivered by the
Alert Generation science payload meet the requirements laid down by \citeds{LSE-61}.

Specifically, this will demonstrate that:

\begin{itemize}

  \item{Difference images have been generated and persisted during
  payload execution;}
  \item{Each difference image includes information about the identity of
	the input exposures, and metadata such as a representation of the
		PSF matching kernel (DMS-REQ-0074);}
  \item{Masks are correctly propagated from the input images.}

\end{itemize}

This test does not include quantitative targets for the science quality criteria; we instead require for each test that we be able to quickly construct a plot or display summary images that allow such a target can be visualized.

\subsubsection{Intercase dependencies}

\begin{itemize}

  \item{\hyperref[ag-00-00]{AG-00-00}}
  \item{\hyperref[ag-00-05]{AG-00-05}}

\end{itemize}

\subsubsection{Environmental needs}

\paragraph{Hardware}

The test shall be carried out on a machine with at least 16\,GB of RAM and
multiple CPU cores which has access to the \texttt{/datasets} shared (GPFS)
filesystem at the LSST Data Facility.

\paragraph{Software}

Release 14.0 of the DM Software Stack will be pre-installed (following the
procedure described in \hyperref[ag-00-00]{AG-00-00}).

\subsubsection{Input specification}

A complete processing of the DECam ``HiTS'' dataset, as defined at
\url{https://dmtn-039.lsst.io/} and
\url{https://github.com/lsst/ap_verify_hits2015}, through the Alert
Generation science payload.

This dataset shall be made available in a standard LSST data repository,
accessible via the ``Data Butler''.

It is not required that all combinations of visit and CCD have been processed
successfully: a number of failures are expected. However, documentation to
describe processing failures should be provided.

\subsubsection{Output specification}

None.

\subsubsection{Procedure}

\begin{itemize}

  \item{The DM Stack shall be initialized using the \texttt{loadLSST} script
  (as described in \hyperref[ag-00-00]{AG-00-00}).}

  \item{A ``Data Butler'' will be initialized to access the repository.}

  \item{For each processed CCD, the difference image 
	  will be retrieved from the Butler, and
  the existence of all components described in \S\ref{ag-00-15-items} will be
  verified.}

  \item{Scripts from the \texttt{pipe\_analysis} package will be run on every visit to check for the presence of data products and make plots}

  \item{Five sensors will be chosen at random from each of two visits and inspected by eye for unmasked artifacts.}

\end{itemize}
