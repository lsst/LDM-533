\subsection{AG-00-05: Alert Generation Produces Required Data Products}
\label{ag-00-05}

\subsubsection{Requirements}

DMS-REQ-0069, DMS-REQ-0010, DMS-REQ-0269, DMS-REQ-0271

\subsubsection{Test items}
\label{ag-00-05-items}

This test will check that the basic data products produced by Alert
Generation are generated by execution of the science payload.

These products will include:

\begin{itemize}

  \item{Processed visit images (PVIs; DMS-REQ-0069);}
  \item{Difference Exposures (DMS-REQ-0010);}
  \item{DIASource catalogs (DMS-REQ-0269);}
  \item{DIAObject catalogs (DMS-REQ-0271);}

\end{itemize}

\subsubsection{Intercase dependencies}

\begin{itemize}

  \item{\hyperref[ag-00-00]{AG-00-00}}

\end{itemize}

\subsubsection{Environmental needs}

\paragraph{Hardware}

The test shall be carried out on a machine with at least 16\,GB of RAM and
multiple CPU cores which has access to the \texttt{/datasets} shared (GPFS)
filesystem at the LSST Data Facility.

\paragraph{Software}

Release 14.0 of the DM Software Stack will be pre-installed (following the
procedure described in \hyperref[ag-00-00]{AG-00-00}).

\subsubsection{Input specification}

A complete processing of the DECam ``HiTS'' dataset, as defined at
\url{https://dmtn-039.lsst.io/} and
\url{https://github.com/lsst/ap_verify_hits2015}, through the Alert
Generation science payload.

This dataset shall be made available in a standard LSST data repository,
accessible via the ``Data Butler''.

It is not required that all combinations of visit and CCD have been processed
successfully: a number of failures are expected. However, documentation to
describe processing failures should be provided.

\subsubsection{Output specification}

None.

\subsubsection{Procedure}

\begin{itemize}

  \item{The DM Stack and Alert Processing packaged shall be initialized 
  as described in \hyperref[ag-00-00]{AG-00-00}.}

  \item{The alert generation processing will be executed using the verification cluster:

  \begin{verbatim}
  python ap_verify/bin/prepare_demo_slurm_files.py

  # At present we must run a single ccd+visit to handle ingestion before 
  # parallel processing can begin
  ./ap_verify/bin/exec_demo_run_1ccd.sh 410915 25

  ln -s ap_verify/bin/demo_run.sl
  ln -s ap_verify/bin/demo_cmds.conf
  sbatch demo_run.sl
  \end{verbatim}

  and any errors or failures reported.}


  \item{A ``Data Butler'' will be initialized to access the repository.}

  \item{For each of the expected data products types (listed in \S\ref{ag-00-05-items})
  and each of the expected units (PVIs, catalogs, etc.), the data product will be
    retrieved from the Butler and verified to be non-empty.}

  \item{\texttt{DIAObjects} are currently only stored in a database, without
  shims to the Butler, so the existence of the database table and its
  non-empty contents will be verified by directly accessing it using
  \texttt{sqlite3} and executing appropriate SQL queries.}

\end{itemize}
