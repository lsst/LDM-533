\subsection{AG-00-00: Installation of the Alert Generation science payload.}
\label{ag-00-00}

\subsubsection{Requirements}

DMS-REQ-0308.

\subsubsection{Test items}

This test will check:

\begin{itemize}

  \item{That the Alert Generation science payload is available for
  distribution from documented channels;}

  \item{That the Alert Generation science payload can be installed on
  LSST Data Facility-managed systems.}

\end{itemize}

\subsubsection{Intercase dependencies}

None.

\subsubsection{Environmental needs}

\paragraph{Hardware}

This test case shall be executed on a developer system at NCSA which serves as
the ``head node'' or otherwise provides access to filesystems shared by the
LSST Verification Cluster (LSST-VC). We assume that this system will be
\texttt{lsst-dev01.ncsa.illinois.edu} and the filesystem will be a GPFS-based
system mounted at \texttt{/software}.

The test also requires access to one LSST-VC compute node.

\paragraph{Software}

All prerequisite packages listed at
\url{https://pipelines.lsst.io/install/prereqs/centos.html} must be available
on the test system and on the LSST-VC compute node.

\subsubsection{Input specification}

No input data is required for this test case.

\subsubsection{Output specification}

The Alert Generation science payload will be made available on a shared
filesystem accessible from LSST-VC compute notes.

\subsubsection{Procedure}

\begin{enumerate}

  \item{Release 14.0 of the LSST Science Pipelines will be installed into the
  GPFS filesystem accessible at \texttt{/software} on \texttt{lsst-dev01}
  following the instructions at
  \url{https://pipelines.lsst.io/install/newinstall.html}.}

  \item{The lsst\_distrib top level package will be enabled:

  \begin{verbatim}
  source /software/lsstsw/stack3/loadLSST.bash
  setup lsst_distrib
  \end{verbatim}
  }

  \item{The ``LSST Stack Demo'' package will be downloaded onto the test
  system from
  \url{https://github.com/lsst/lsst_dm_stack_demo/releases/tag/14.0} and
  uncompressed.}

  \item{The demo package will be executed by following the instructions in its
  ``README`` file. The string ``Ok.`` should be returned.  
  Specifically, we execute:
  \begin{verbatim}
  setup obs_sdss
  ./bin/demo.sh
  python bin/compare expected/Linux64/detected-sources.txt
  \end{verbatim}
  }

  \item{A shell on an LSST-VC compute node will now be obtained by executing:

  \begin{verbatim}
  $ srun -I --pty bash
  \end{verbatim}
  }

  \item{The demo package will be executed on the compute node and the same
  result obtained.}

  \item{The Alert Production datasets and packages are not yet part of \texttt{lsst\_distrib} and so must be installed separately.   They will be installed as follows on the GPFS filesystem:

  \begin{verbatim}
  setup git_lfs
  git clone https://github.com/lsst/ap_verify_hits2015.git

  export AP_VERIFY_HITS2015_DIR=$PWD/ap_verify_hits2015
  cd $AP_VERIFY_HITS2015_DIR
  setup -r .
  cd -

  setup obs_decam

  git clone https://github.com/lsst-dm/ap_association
  cd ap_association
  setup -k -r .
  scons 
  cd - 

  git clone https://github.com/lsst-dm/ap_pipe
  cd ap_pipe
  setup -k -r .
  scons 
  cd - 

  git clone https://github.com/lsst-dm/ap_verify
  cd ap_verify
  setup -k -r .
  scons 
  cd - 
  \end{verbatim}

  and any errors or failures reported.}

\end{enumerate}
