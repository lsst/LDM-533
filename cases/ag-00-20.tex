\subsection{AG-00-20: Scientific Verification of DIASource Catalog}
\label{ag-00-20}

\subsubsection{Requirements}

DMS-REQ-0269, DMS-REQ-0270, DMS-REQ-0347, DMS-REQ-0331.

\subsubsection{Test items}
\label{ag-00-20-items}

This test will check that the difference image source catalogs
delivered by the Alert Generation science
payload meet the requirements laid down by \citeds{LSE-61}.

Specifically, this will demonstrate that:

\begin{itemize}

  \item{Measurements in the catalog are presented in flux units (DMS-REQ-0347);}

  \item{Each DIASource record contains an appropriate subset of the attributes
  required by DMS-REQ-0269. In particular, the LDM-503-3-era pipeline is
  expected to provide DIASource positions (sky and focal plane), fluxes, and
  flags indicative of issues encountered during processing.}

  \item{Faint DIASources satisfying additional criteria are stored (DMS-REQ-0270).}

  \item{Derived quantities are provided in pre-computed columns (DMS-REQ-0331);}

\end{itemize}

This test does not include quantitative targets for the science quality criteria.

\subsubsection{Intercase dependencies}

\begin{itemize}

  \item{\hyperref[ag-00-00]{AG-00-00}}
  \item{\hyperref[ag-00-05]{AG-00-05}}

\end{itemize}

\subsubsection{Environmental needs}

\paragraph{Hardware}

The test shall be carried out on a machine with at least 16\,GB of RAM and
multiple CPU cores which has access to the \texttt{/datasets} shared (GPFS)
filesystem at the LSST Data Facility.

\paragraph{Software}

Release 14.0 of the DM Software Stack will be pre-installed (following the
procedure described in \hyperref[ag-00-00]{AG-00-00}).

\subsubsection{Input specification}

A complete processing of the DECam ``HiTS'' dataset, as defined at
\url{https://dmtn-039.lsst.io/} and
\url{https://github.com/lsst/ap_verify_hits2015}, through the Alert
Generation science payload.

This dataset shall be made available in a standard LSST data repository,
accessible via the ``Data Butler''.

It is not required that all combinations of visit and CCD have been processed
successfully: a number of failures are expected. However, documentation to
describe processing failures should be provided.

\subsubsection{Output specification}

None.

\subsubsection{Procedure}

\begin{itemize}

  \item{The DM Stack shall be initialized using the \texttt{loadLSST} script
  (as described in \hyperref[ag-00-00]{AG-00-00}).}

  \item{A ``Data Butler'' will be initialized to access the repository.}

  \item{DIASource records will be accessed by querying the Butler, then
  examined interactively at a Python prompt.}

\end{itemize}
