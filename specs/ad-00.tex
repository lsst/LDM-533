\subsection{AD-00: Alert Distribution Testing}
\label{ad-00}

\subsubsection{Objective}

This test specification demonstrates the successful execution of the major components of the Alert Distribution system at the scale of LSST Operations.

It will demonstrate that:

\begin{itemize}

  \item{The Alert Distribution payloads 
	  can be made available on systems managed by the
  LSST Data Facility;}

  \item{The Alert Distribution payloads can be executed under the
  control of the Prompt Processing service;}

  \item{All required science data products can be passed through the components
	  of the alert distribution service.}

\end{itemize}

Note that this test specification does not extend to 
testing integration of the Alert Distribution System with the Alert Generation Science Pipe or the LSST Science Platform. 
We also do not test the Alert Database at this time as its functional requirements are not sufficiently specified to enable testing.

\subsubsection{Approach refinements}

The general approach defined in \citeds{LDM-503} is used.

\subsubsection{Test case identification}

\begin{longtable} {|p{0.4\textwidth}|p{0.6\textwidth}|}\hline
\textbf{Test Case}  & \textbf{Description} \\\hline

\hyperref[ad-00-00]{AD-00-00} & Tests that the Alert Distribution payloads can be installed and executed on LSST Data Facility-managed systems.\\\hline
\hyperref[ad-00-05]{AD-00-05} & Tests that alert packets can be loaded and extracted from the queue system. \\\hline
\hyperref[ad-00-10]{AD-00-10} & Tests that alert packets can be filtered by the mini-broker system. \\\hline
\end{longtable}
